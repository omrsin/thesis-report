\TUsection{Java PathFinder}
\label{sec:jpf}

% (Definition of JPF) (verification, analysis and testing environment for Java, called Java PathFinder (JPF)) (JPF is a highly customizable execution environment for verification of Java™ bytecode programs)

% (Java, Mention and reference it) THE Java® programming language is a general-purpose, concurrent, class-based, object-oriented language. It is designed to be simple enough that many programmers can achieve fluency in the language. The Java programming language is strongly and statically typed. This specification clearly distinguishes between the compile-time errors that can and must be detected at compile time, and those that occur at run time.

%(JPF Motivation Why analyze code and in particular Java) We believe it is an attractive idea to develop a verification environment for Java for three reasons. First, Java is a modern language featuring important concepts such as object-orientation and multi-threading within one language. Second, Java is simple, for example compared to C++. Third, Java is compiled into bytecode, and hence, the analysis can be done at the bytecode level. This implies that such a tool can be applied to any language that can be translated into bytecode.

%(Why Java --- It is well known that concurrent programs are non-trivial to construct, and with Java essen- tially giving the capability for anyone to write concurrent programs, we believe, a model checker for Java might have a bright future. In fact, one area where we believe it can have an immediate impact is in environments where Java is taught. Also because it is widely used.)

%(Maybe not necessary but valuable to explain native calls --- The JVMJPF supports all Java bytecodes, hence any program written in pure Java can be analyzed. Unfortunately, not all Java programs consist of pure Java code—one often finds that certain methods are defined as being native to the operating system. When a Java program calls methods that have no corresponding bytecodes, then JPF cannot determine what the state of these code fragments will be and hence cannot handle programs that, for example, access the file system (user-defined class- loaders, file I/O operations, etc.), or communicate over a network, contains GUI code, etc. Fortunately, many native methods do not have side-effects and hence simple wrapper- methods can be written that translate the inputs and outputs to the native method, which then allow the original method to be called and all state changes to happen after returning from the call)

%(With JPF we took the other route, we developed our own custom-made model checker that can execute all the bytecode instructions, and hence allow the whole of Java to be model checked. The model checker consists of our own Java Virtual Machine (JVMJPF) that executes the bytecodes and a search component that guides the execution. It is an explicit state model checker!!!)

%(Features of JPF) (Explicit State, State Matching, Backtracking, Partial Order Reduction)

%(States --- Specifically, each state consists of three components: information for each thread in the Java program, the static variables (in classes) and the dynamic variables (in objects) in the system. The information for each thread consists of a stack of frames, one for each method called, whereas the static and dynamic information consists of information about the locks for the classes/objects and the fields in the classes/object)

%(State is a snapshot of the current execution status of the application (mostly thread and heap states), plus the execution history (path) that lead to this state. Every state has a unique id number. JPF-internally, State is encapsulated in the SystemState instance (almost, there is some execution history which is just kept by the JVM object). This includes three components:
%KernelState - the application snapshot (threads, heap)
%trail - the last Transition (execution history)
%current and next ChoiceGenerator - the objects encapsulating the choice enumeration that produces different transitions (but not necessarily new states))

%(Termination --- In order to ensure termination during explicit state model checking one must know when a state is revisited. It is common for a hashtable to be used to store states, which means an efficient hash function is required as well as fast state comparison.)

%(A major design decision for JPF was to make it as modular and understandable to others as possible, but we sacrificed speed in the process—Spin is at least an order of magnitude faster than JPF.We believe this is a price worth paying in the long run.)

%(Extension Points: Search strategies, Choice Generators, Listeners, Instruction Factories, Native Peers, Models, Publisher, consider presenting them as a list of concepts)

Developed at \acrshort{acr:nasa}'s Ames Research Center~\cite{WebNASAAmes2017}, \acrfull{acr:jpf} is an execution environment for verification and analysis of Java bytecode programs~\cite{Visser2003,WebJPF2017}. Since its publication in the year 2000~\cite{Havelund2000}, \jpf has evolved from being a model translator to a fully fledged, highly customizable virtual machine capable of controlling and augmenting the execution of a program.

Java is a widely known, general-purpose programming language with strong roots on concurrency support and object-oriented principles~\cite{Gosling2014}. Programs written in Java are compiled to the standardized instruction set of the \acrfull{acr:jvm}, known as Java bytecode. This process enables Java programs to be portable between architectures implementing the \acrshort{acr:jvm} specification. A \acrshort{acr:jvm} implementation serves as an interpreter of Java bytecode and allows the optimization and execution of the program tailored for the host platform~\cite{Lindholm2014}.

\jpf focuses on Java mainly for three reasons: its wide adoption as a modern programming language, its simplicity in comparison to other high profile languages, and the flexibility in terms of bytecode analysis; potentially enabling the verification of any other language capable of being compiled into Java bytecode. Moreover, the non-trivial nature of concurrent programs makes them difficult to construct and debug. A model checker with the capacity of validating concurrent Java programs is crucial for ensuring correctness of mission-critical software, such as the likes required by \acrshort{acr:nasa}.

In its core, \jpf is a \acrlong{acr:jvm} implemented in Java itself, comprised of several extensible modules that dictate the verification strategy to be followed. The fact that \jpf is written in Java means that it is executed on a canonical \acrshort{acr:jvm}; in other words, a \acrshort{acr:jvm} on top of a \acrshort{acr:jvm}. 

\begin{figure}[t]
\centering
\includegraphics[height=5cm]{example-image}
\caption{\jpf Workflow}
\label{fig:jpf:process}
\end{figure}

% TODO: Add an entry to the glossary for explicit state model checking, state, space state, state explosion and maybe partial-order reduction
By default, \jpf's mode of operation is \textit{explicit-state model checking}. This means that \jpf keeps track of the execution status of a program, commonly referred to as a state, to check for violation of predefined properties. A state is characterized by the information of existing threads, the contents of the heap, and the sequence of previous states that led to the current execution point (also known as path). A change in any of the aforementioned aspects represents a transition to a new state. Additionally, \jpf associates complementary information to a state (e.g., range of possible values that trigger transitions), in order to reduce the total number of states to be explored. Termination is ensured by avoiding revisiting states.

Figure \ref{fig:jpf:process} portrays the components that participate in a verification process using \jpf. The program under test is loaded into \jpf's core, where its instructions are executed one by one until an execution choice is found. At this point, \jpf records the current state and attempts to resume execution, exploring all possible scenarios based on the choice criteria. Once a chosen path has been completely explored, \jpf backtracks to a recorded state, in order to explore a new path.

\begin{lstlisting}[
language=Java,
caption={[Simple example with random values] The use of random values could lead to unexpected behavior. In this case, a division by zero could occur if certain combinations of random values are used.},
float,
label=lst:jsp:random
]
import java.util.Random;

public class RandomExample {
  public static void main(String[] args) {
    Random random = new Random();
    int a = random.nextInt(2);
    int b = random.nextInt(3);
    int c = a/(b+a-2);
  }
}
\end{lstlisting}

%(Simple example) (Division by zero seems reasonable)

\begin{figure}[t]
\centering
\includegraphics[height=5cm]{example-image}
\caption{State space of the random example}
\label{fig:jpf:random}
\end{figure}

A key aspect of \jpf was to make it extensible and customizable. With a modular design, users of the tool are capable of tuning \jpf to the needs of a wide variety of analyses and verifications. The main extension points are:

\begin{itemize}
	\item \textbf{Bytecode Factories}: Define the semantics of the instructions executed by \jpf's virtual machine. Modifications to the bytecode factory define de execution model of the analyzed program (e.g., operations on symbolic values).
	
	\item \textbf{Choice Generators}: For every path exploration, a set of possible choices must be provided in order to explore different behaviors of the system under test (e.g., a range of integer values for validation of random input). This aspect is critical reduce the number of states explored during a validation, hence scoping the reach of an analysis.
	
	\item \textbf{Listeners}: Serve as monitoring points for interacting with the execution of \jpf. Listeners react to particular events triggered during the execution of an analysis, providing the right environment for the assertion of different properties.
	
	\item \textbf{Native Peers}: In some cases, a system under test will contain calls that are irrelevant to the analysis carried out (e.g., calling external libraries) or will execute native instructions that cannot be interpreted by \jpf. For these cases, native peers provide a mechanisms for modeling the behavior of such situations and efficiently delegating their execution to the host virtual machine.
	
	\item \textbf{Publishers}: Report the outcome of an analysis. Whether a property was violated or the system under test was explored satisfactorily, publisher provide the information that makes the analysis valuable.   
	
	\item \textbf{Search Strategies}: Indicate how the state space of the system under test is to be explored. In other words, the search strategy tells \jpf when to move forward and generate a new state or when to backtrack to a previously known state in order to try a different choice. Search strategies can be customized to guide the exploration of the state space to areas of interests where the analysis is most likely to detect an anomaly. 
\end{itemize}

(Modules - Briefly mention them a an introduction to SPF)

\TUsubsection{Symbolic PathFinder (SPF)}
\label{subsec:spf}

(Definition of SPF)

(Explain its extension points: Choice Generators, Listeners, Symbolic Instruction Factory)

(Mention the solvers)







