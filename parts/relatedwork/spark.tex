\TUsection{Apache Spark}
\label{sec:spark}

%TODO How to include somewhere that Spark was built in Scala \cite{WebScala2017}

Spark is a distributed data processing framework that was first introduced in 2012~\cite{Zaharia2012a}. Similar to other systems, such as MapReduce~\cite{Dean2008} and Dryad~\cite{Isard2007}, it aims to provide a clean and flexible abstraction to distributed computations on large datasets. However, Spark offers two advantages in comparison to such systems: It makes use of a shared memory abstraction that improves performance by avoiding persisting intermediate sets. It also maintains an efficient fault-tolerance mechanism, based on tracking coarse-grained operations, that can recover lost tasks with minimal impact.

The working units in Spark are called \textit{\acrlongpl{acr:rdd}}, better known as \acrshortpl{acr:rdd}. These units represent an immutable partitioned collection of elements in a distributed memory space. \acrshortpl{acr:rdd} can only be created through a set of deterministic operations, known as \textit{transformations} (e.g., \textit{map}, \textit{filter} and \textit{join}), that can be applied to both, raw data or other \acrshortpl{acr:rdd}. Transformations are not evaluated immediately, instead Spark keeps track of all the transformations applied to each \acrshort{acr:rdd} in a program so it can optimize their subsequent processing. Additionally, \acrshortpl{acr:rdd} can be made persistent into storage or can be operated to produce a value. This kind of operations are known as \textit{actions} (e.g., \textit{count}, \textit{reduce} and \textit{save}), and they are the ones that trigger the processing of \acrshortpl{acr:rdd}.

\begin{lstlisting}[
	language=Scala,
	float,
	caption={[Log processing with Spark] Entries in a log file are filtered, grouped and counted based on a common property. Finally the result is saved to persistent storage.},
	label=lst:spark:simple
]
val log = spark.textFile("*file*") /*# \label{lst:ln:spark:conf} #*/
val errors = log.filter(_.contains("ERROR")) /*# \label{lst:ln:spark:transformations-begin} #*/
	.map(error => (error.split('\t')(0),1))
	.reduceByKey(_+_) /*# \label{lst:ln:spark:transformations-end} #*/
errors.save() /*# \label{lst:ln:spark:actions} #*/
\end{lstlisting}

To interact with the \acrshort{acr:rdd} abstraction, Spark provides several \acrshortpl{acr:api} for different programming languages such as Java, Scala, Python and recently R~\cite{Venkataraman2016}. Listing~\ref{lst:spark:simple} presents a simple Spark program written with the Scala \acrshort{acr:api}, that processes log files in the search for errors. The operation in line~\ref{lst:ln:spark:conf} creates the first \acrshort{acr:rdd} from a log file, whose origin could be a local file or a partitioned file in a distributed file system such as \acrfull{acr:hdfs}~\cite{WebHadoop2017}. Spark converts each line in the file to a \textit{String} element in the newly created \acrshort{acr:rdd}. In lines~\ref{lst:ln:spark:transformations-begin} to~\ref{lst:ln:spark:transformations-end}, a chain of transformations is applied to the \acrshort{acr:rdd}: First, elements not containing the text ``ERROR'' are filtered. Next, the remaining elements are transformed to tuples consisting of a certain property (e.g., a time stamp; assumed to be the first information in a log entry) and the number~1. Finally, the tuples are grouped and counted based on the chosen property. Line~\ref{lst:ln:spark:actions} represents the action applied to the \acrshort{acr:rdd}, in this case, saving it to persistent storage.

% TODO: Glossary entries: Spark, RDD, Distributed File System, data locality, lineage
During the execution of a program, Spark does not generate imperatively new data collections for every transformation it finds. Instead, it constructs new \acrshortpl{acr:rdd} attached with the operation that has to be applied to each element. The resulting \acrshort{acr:rdd} is a sequence of operations starting from the source dataset, whose semantics depends on the nature of each transformation involved. It is not until an action is found that the target \acrshort{acr:rdd} is resolved and the whole sequence of transformations actually operates the data. 

Delaying the resolution of \acrshortpl{acr:rdd} in this way allows Spark to improve the distribution of operations in a clustered dataset, taking advantage of properties like data locality. Moreover, the trace of operations that produced a certain element in an \acrshort{acr:rdd}, known as \textit{\gls{gls:lineage}}, enables Spark to recover failed tasks only recalling to the necessary data elements that reproduce the lost portion. Figure~\ref{fig:spark:lineage} depicts the resulting \gls{gls:lineage} of the program explained in listing \ref{lst:spark:simple} and figure~\ref{fig:spark:execution} shows a sample execution of the same program.

% TODO: Establish a default for the captions of figures and add a figure to the lineage

% Define block styles
\tikzstyle{block} = [rectangle, draw, fill=white, text centered, rounded corners, minimum height=3em, minimum width=5em, drop shadow]
\tikzstyle{block-data} = [rectangle, draw, fill=white, align=left,rounded corners, minimum height=3em, minimum width = 5em, drop shadow]
\tikzstyle{line} = [draw, -{>[length=2mm, width=1mm]}]

\begin{figure}[t]
\centering
\begin{tikzpicture}[node distance = 5cm, auto]
	% Place nodes
	\node [block] (INPUT) {log};
	\node [block, right of=INPUT] (ERROR) {errors};
	\node [block, right of=ERROR] (PAIRS) {pairs};
	\node [block, right of=PAIRS] (REDUCE) {sum};
	% Draw edges
	\path [line] (INPUT) -- node{\textit{filter}}(ERROR);
	\path [line] (ERROR) -- node{\textit{map}}(PAIRS);
	\path [line] (PAIRS) -- node{\textit{reduceByKey}}(REDUCE);
\end{tikzpicture}

\begin{minipage}[t]{\linewidth}
	\subcaption{\Gls{gls:lineage} of the program shown in listing~\ref{lst:spark:simple}. After each transformation, a new node in the lineage tree is created.} 
	\label{fig:spark:lineage}
\end{minipage}
\vspace{\belowdisplayskip}

\begin{tikzpicture}[node distance = 4.7cm, auto]
	% Place nodes
	\node [block-data, below of=INPUT] (INPUTDATA) {\small \texttt{2017/02 ERROR \ldots} \\ 
		\small \texttt{2017/02 INFO ~\ldots} \\
		\small \texttt{2017/03 ERROR \ldots} \\
		\small \texttt{2017/03 ERROR \ldots}};
	\node [block-data, right of=INPUTDATA] (ERRORDATA) {\small \texttt{2017/02 ERROR \ldots} \\ 
		\small \texttt{2017/03 ERROR \ldots} \\
		\small \texttt{2017/03 ERROR \ldots}};
	\node [block-data, right of=ERRORDATA] (PAIRSDATA) {\small \texttt{(2017/02, 1)} \\ 
		\small \texttt{(2017/03, 1)} \\
		\small \texttt{(2017/03, 1)}};
	\node [block-data, right of=PAIRSDATA] (REDUCEDATA) {\small \texttt{(2017/02, 1)} \\ 
		\small \texttt{(2017/03, 2)}};
	% Draw edges
	\path [line] (INPUTDATA) -- (ERRORDATA);
	\path [line] (ERRORDATA) -- (PAIRSDATA);
	\path [line] (PAIRSDATA) -- (REDUCEDATA);
\end{tikzpicture}
\begin{minipage}[t]{\linewidth}
	\subcaption{Sample execution of the program shown in listing~\ref{lst:spark:simple}. If a task failed, Spark is capable to recalculate only the missing portions by retracing the operations in the \gls{gls:lineage} that led to the missing data.}
	\label{fig:spark:execution}
\end{minipage}

\caption[\Gls{gls:lineage} tree and execution of a Spark program]{\Gls{gls:lineage} and execution of the Spark program shown in listing~\ref{lst:spark:simple}. The \gls{gls:lineage} is independent from the association of an \acrshort{acr:rdd} to a variable; for example, the \acrshort{acr:rdd} resulting from the filter transformation is not assigned to a variable, however it is a node in the lineage tree.}
\end{figure}

% TODO: Glossary entries: higher-order functions, user-defined aggregations, closure
Most of the operations in Spark are higher-order functions, this means they accept one or more functions as parameters. For example, the \textit{filter} transformation requires a function that takes an element of the \acrshort{acr:rdd} and evaluates to a boolean value. These user-defined functions work as closures by scoping their environment even if it contains references to variables outside itself; this enables Spark to ensure consistency when applying such functions in parallel nodes. The use of higher-order functions serve as a flexible mechanism to adapt Spark's computation model to different tasks.

% TODO: Maybe include a referece to machine learning algorithms that are iterative (this could be considered common knowledge)
The inherent capacity of Spark to operate in a distributed memory space makes it well-suited for two particular scenarios: iterative algorithms and interactive querying. The former, which are commonplace among machine learning algorithms, leverages on the reuse of datasets and avoids having to perform costly I/O operations for every iteration. The latter, allows data mining techniques to synthesize queries faster by keeping working data at hand.

Spark is part of the Apache Software Foundation and it is offered as an open-source software~\cite{WebApache2017,WebSpark2017}. Several purpose-specific libraries are built on top of Spark, as is the case of: MLlib for machine learning~\cite{Meng2016}, GraphX for graph computations~\cite{Xin2013}, Spark Streaming for stream processing~\cite{Zaharia2013}, and Spark SQL, an \mbox{SQL-like} interface for structured querying in Spark~\cite{Armbrust2015a}.

In 2014, Spark reported the fastest Daytona GraySort as defined by the Sort Benchmark committee, and later in 2016, Spark was part of the technology stack that claimed the most resource-efficient Daytona CloudSort as defined by the same committee~\cite{WebSortBenchmark2017,Xin2014,Wang}. Overall, Spark offers a better performance in comparison to other data processing frameworks.

%(Version: Be sure to mention the current version of Spark at the moment or the version used for the analysis)
%
%(Transformations and Actions supported in paper) Table 2: Transformations and actions available on RDDs in Spark. Seq[T] denotes a sequence of elements of type T \cite{Zaharia2012a}
%
%(Graph representation) We propose a simple graph-based representation for RDDs that facilitates these goals. (The goal of connecting consecutive RDD operations) \cite{Zaharia2012a}
%
%(Concept of narrow dependency. Maybe not useful here but I don't want to forget) First, narrow dependencies allow for pipelined execution on one cluster node, which can compute all the parent partitions. For example, one can apply a map followed by a filter on an element-by-element basis \cite{Zaharia2012a}
%
%(Runs under Mesos maybe not necessary) The system runs over the Mesos cluster manager, allowing it to share resources with Hadoop, MPI and other applications \cite{Zaharia2012a}
%
%Check R. Bose and J. Frew. Lineage retrieval for scientific data processing: a survey and Cheney, L. Chiticariu, andW.-C. Tan. Provenance in databases: Why, how, and where