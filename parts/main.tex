\TUchapter{Symbolic Execution of Spark Programs}
\label{ch:symbolic-spark}

To provide a flexible programming paradigm for big data processing, Spark's main \acrshort{acr:api} exposes methods that act as higher order functions. Particularly, these methods receive user defined functions as input parameters that dictate how certain operations will be carried out. However, the passed functions always have to comply to some conditions imposed by the method, for example, the function passed to a \textit{filter} transformation must be defined over the data type of the target \acrshort{acr:rdd} and must return a \textit{boolean} value.

The use of functions as parameters in Spark operations has an impact on the control flow of the program. Not only the particular Spark operation defines how the program will behave, but also the passed functions could potentially introduce control flow statements like conditionals or loops. Moreover, the control flow behavior of the Spark operations themselves is mostly static (e.g., the iterative and cumulative nature of a \textit{reduce} action), whereas the diverse range of variation introduced by a user defined function is practically unbounded.

For this reason, in the context of program analysis in general and to our particular scope of symbolic execution, both the nature of the Spark operations and the peculiarities of the user defined functions on each program are necessary components that have to be studied together in order to provide a reasonable conclusion.

The following sections describe the conceptual process of a symbolic execution carried out on a Spark program, as well as a detailed explanation of our proposed implementation.

\TUsection{Conceptual Process}
\label{sec:conceptual-process}

% Mention the structure of spark programs (series of transformations culminating in an action)

% Mention that RDD transformations themselves are not the targets of the analysis rather the invocation of the passed functions as arguments

% Create a diagram with a chain of transformations and the potential changes in the input and how this operations can include conditionals

% Mention the special case of filters that are conditionals by themselves

% Mention the special case of flatMap and their increased cardinality in the output

A Spark program consists of a chain of transformations on one or more \acrshort{acr:rdd}s that finally conclude with an action applied on them. \acrshort{acr:rdd}s are manipulated through an \acrshort{acr:api} that provides the general guidelines on how the data collection is to be processed without falling into the specifics. For example, the \textit{filter} transformation indicates that only the elements matching a given filtering condition would be selected, without specifying exactly what is the condition to be evaluated. A similar approach is followed by most of the actions and transformations in Spark. Many of these operations come from functional programming languages and define abstractions to interact with data collections.
%TODO: MAYBE Refer to the first part of the diagram

The precise behavior of most operations is defined by the programmer. Given that most of Spark's actions and transformations are higher order functions, the programmers define a custom function that fulfills the contract of the specific operation. For example, again the \textit{filter} transformation expects as a parameter a function that takes an element of the same type as the type of the elements in the collection handled by the \acrshort{acr:rdd} and returns a boolean value. When the \textit{filter} transformation is later invoked, it calls the passed function with each element in the \acrshort{acr:rdd} and, depending on the output, it decides if the value is filtered or not.

%TODO: MAYBE This are two linear diagrams to illustrate a chain of transformations and underneath a chain of calls to their respective functions bound with conditions particular to eachs transformation.

%% Define block styles
%\tikzstyle{block} = [rectangle, draw, fill=white, align=center,rounded corners, minimum height=3em, minimum width=3em, drop shadow]
%\tikzstyle{line} = [draw, -{>[length=2mm, width=1mm]}]
%\begin{figure}[t]
%	\centering
%	\begin{tikzpicture}[node distance = 5cm, auto]
%	\end{tikzpicture}
%	
%	\begin{minipage}[t]{\linewidth}
%	\end{minipage}
%	\vspace{\belowdisplayskip}
%	
%	\begin{tikzpicture}[node distance = 4.7cm, auto]
%	\end{tikzpicture}
%	
%	\begin{minipage}[t]{\linewidth}
%	\end{minipage}
%	
%	\caption[*]{*}
%\end{figure}

Having this in mind, the symbolic execution of an isolated Spark operation depends solely on the behavior of the function passed by the user. However, when analyzing a whole program, special considerations for every particular operation must be taken into account. These considerations are different in nature but mostly refer to how output values are percolated to the subsequent operations in order to ensure the correct analysis of the next functions.
%TODO: MAYBE Refer to the second part of the diagram

The whole process could be summarized in the following three steps:
\begin{enumerate}
	\item Identify the Spark operation
	\item Carry out the symbolic execution of the passed function
	\item Take special considerations based on the executed Spark operation
\end{enumerate}

Following these three steps, the symbolic execution of a Spark program, using \spf{} as the underlying analysis framework, is represented by the state diagram depicted in~\ref{fig:logic:state-diagram}. Here we consider how a black-box analysis should proceed in order to reason about the execution flow of a Spark program and the control flow instructions that might occur in it.

The starting point of the analysis consists in detecting that a Spark operation of interest is being executed; relevant operations can be defined by the user beforehand. Once this has happened, the next step is to prepare for the exact operation that was detected, for example, indicating what function was passed to the detected operation and prepare the \spf{} analysis to consider its input parameters as symbolic values. Generally, these two steps occur simultaneously but given their semantic differences in the process, it was important to highlight them as different states of the analysis.

The real analysis begins once the passed function is invoked. The process is split in three stages: \textit{pre-process}, \textit{analysis} and \textit{post-process}. During the \textit{pre-process} stage, we must ensure that the parameters passed to the function are correctly instantiated; for example, if a \textit{map} and a \textit{filter} transformations took place in that order, we must ensure that the input symbolic expression passed to the function executed by the \textit{filter} is the output of the function invoked by the \textit{map} transformation. This guarantees a coherent inter-methodical analysis. The subsequent stage is the core analysis, which proceeds in the same way as an analysis of a regular method in \spf{} would do. Lastly, during the \textit{post-process} stage the framework makes all the necessary preparations to be able to continue the symbolic execution of subsequent Spark operations.

Once the analysis of a Spark operation is done, the framework continues to explore the program. This can lead to the detection of another relevant Spark operation. Finally, once the execution has finished, \jpf{} will backtrack to any decision points defined by the \textit{Choice Generators}. These points always take place inside one of the functions passed to any Spark operation; this is why the framework has to re-establish a strategy corresponding the Spark operation containing the invoked function. The analysis continues as usual once the strategy has been re-established. After all choices have been explored the execution terminates and the analysis provides an output.

% Define block styles
\tikzstyle{block} = [rectangle, draw, fill=white, align=center,rounded corners, minimum height=3em, minimum width=3em, drop shadow]
\tikzstyle{line} = [draw, -{>[length=2mm, width=1mm]}]
\begin{figure}[t]
	\centering
	\begin{tikzpicture}[node distance = 2.5cm, auto]
	% Place nodes
	\node [block, very thick] (DETECTSPARK) {\small Detect \\ \small Spark \\ \small Operation};
	\node [block, right of=DETECTSPARK] (PREPARESTRATEGY) {\small Prepare \\ \small Strategy};	
	\node [block, right of=PREPARESTRATEGY] (DETECTFUNC) {\small Detect \\ \small Function \\ \small Invocation};
	\node [block, right of=DETECTFUNC] (PREPROC) {\small Pre-process};
	\node [block, right of=PREPROC] (ANALYSIS) {\small Analyze};
	\node [block, below of=ANALYSIS] (POSTPROC) {\small Post-process};
	\node [block, right of=ANALYSIS] (BACKTRACK) {\small Backtrack};
	\node [block, above of=BACKTRACK] (RESTABLISHSTRATEGY) {\small Re-establish \\ \small Strategy};
	\node [block, right of=BACKTRACK, double distance=1pt] (TERMINATE) {\small Terminate \\ \small Execution};			
	% Draw edges
	\path [line, rounded corners] (DETECTSPARK) -- (PREPARESTRATEGY);
	\path [line, rounded corners] (PREPARESTRATEGY) -- (DETECTFUNC);
	\path [line, rounded corners] (DETECTFUNC) -- (PREPROC);
	\path [line, rounded corners] (PREPROC) -- (ANALYSIS);
	\path [line, rounded corners] (ANALYSIS) -- (POSTPROC);
	\path [line, rounded corners] (POSTPROC) -| (BACKTRACK);
	\path [line, rounded corners] (POSTPROC) -| (DETECTSPARK);
	\path [line, rounded corners] (BACKTRACK) -- (RESTABLISHSTRATEGY);
	\path [line, rounded corners] (RESTABLISHSTRATEGY) -| (ANALYSIS);
	\path [line, rounded corners] (BACKTRACK) -- (TERMINATE);
	\end{tikzpicture}
	\caption[State Diagram of the Symbolic Execution Process of Spark Programs]{State diagram of the symbolic execution process of Spark programs.}
	\label{fig:logic:state-diagram}
\end{figure}

\TUsection{Concrete Examples}
\label{sec:logic:example}

Word count~\cite{Dean2008} has become the de facto example used in studies related to big data frameworks and distributed processing. We present a modified version of the regular word count example that allows us to illustrate the use case of symbolic execution in Spark programs. The goal of the word count algorithm is to process a document or group of documents to determine how many times each word appeared in the analyzed data. The modification we introduce puts a restriction on the structure of the words that are counted; only words that begin with a certain prefix are considered in the algorithm. This modification permits the existence of multiple paths while still keeping the original notion of the word count example. 

\begin{lstlisting}[
language=Java,
caption={[Selective Word Count Example] A modified version of the word count algorithm. Only words that begin with certain prefixes are taken into consideration.},
float=h,
label=lst:logic:word-count-example
]
JavaRDD<String> textFile = sc.textFile("hdfs://...");
JavaPairRDD<String, Integer> counts = textFile
	.flatMap(s -> Arrays.asList(s.split(" ")).iterator())
	.filter(x -> x.startsWith("re") || x.startsWith("un") || x.startsWith("in")) /*# \label{lst:ln:logic:word-count-example:filter} #*/	
	.mapToPair(word -> new Tuple2<>(word, 1))
	.reduceByKey((a, b) -> a + b);	
counts.saveAsTextFile("hdfs://...");
\end{lstlisting}

Listing~\ref{lst:logic:word-count-example} shows how the selective word count algorithm is implemented in Spark. The algorithm proceeds as expected; the document is split into composing words and later these words are grouped and counted. The only difference is introduced in line~\ref{lst:ln:logic:word-count-example:filter}, where an additional \textit{filter} transformation is included after the document has been split into words. This operation will filter out all words except those who begin with the ``re'', ``un'', and ``in'' prefixes. The only condition imposed over the structure of the input data is introduced by this action, hence it is only relevant to analyze the \textit{filter} transformation in order to determine all possible execution paths.

Only two words would suffice to explore all the possible paths of the program; one that would start with the aforementioned prefixes and one that would not. However, in order to try out all the conditions in the disjunction, at least one word matching each prefix would be necessary. Matching each condition in the disjunction is necessary in those cases where the boolean expression defining the path condition has to be thoroughly evaluated. Figure~\ref{fig:logic:symbolic-spark-word-count-example} shows the symbolic execution tree of the selective word count algorithm. Note that this execution tree depicts an exhaustive exploration of the boolean expression, thus selecting the most simple satisfying word for each condition in the disjunctions (in this case, the prefix itself).

\begin{figure}[t]
	\[\xymatrix@C=0.3em @R=0.8em{ 
		& & & & & \ar[dd]^<(0.4){x = V_{0}} filter & & \\
		& & & & & & & \\
		& & & & & \ar@{}[llllldddd]_<(0.2){true} \ar[llllldddd]
		\ar@{}[lldddd]_<(0.2){true} \ar[lldddd]
		\ar@{}[dddd]_<(0.2){true} \ar[dddd]
		\txt{\small $~~V_{0}.startsWith(``re")$ \\ 
			\small $\lor ~V_{0}.startsWith(``un")$ \\
			\small $\lor ~V_{0}.startsWith(``in")$} 
		\ar[rrdddd] \ar@{}[rrdddd]^<(0.2){false} & & \\
		& & & & & & & \\
		& & & & & & & \\
		& & & & & & & \\
		*+[F] \txt{\small $V_{0}.startsWith(``re")$} & & &
		*+[F] \txt{\small $V_{0}.startsWith(``un")$} & & 
		*+[F] \txt{\small $V_{0}.startsWith(``in")$} & &
		*+[F] \txt{\small $~~\lnot(V_{0}.startsWith(``re")$ \\ 
			\small $\lor ~V_{0}.startsWith(``un")$ \\
			\small $\lor ~V_{0}.startsWith(``in"))$} \\
		V_{0} = ``re" & & & V_{0} = ``un" & &  V_{0} = ``in" & & V_{0} = ``\_" \\
	} \]
	\caption[Symbolic Execution Tree of the Selective Word Count Example]{Symbolic execution tree corresponding to the Spark program shown in listing~\ref{lst:logic:word-count-example}. An input file containing the words $``re",``un",``in"$ and $``\_"$ would suffice to explore all feasible paths in the program.}
	\label{fig:logic:symbolic-spark-word-count-example}
\end{figure}


The selective word count algorithm is a good example to convey the usefulness of symbolic execution analyses in the context of big data programs. However, it is not capable of illustrating some particular considerations given the lack of connected constraints between operations. 

The trivial example presented in listing~\ref{lst:logic:example} depicts a simple Spark program with no purpose in itself. Nonetheless, this simple example allows us to better explain how the analysis will be carried out when several relevant operations are present in the program under test. The relevant Spark operations in this example are the \textit{map} and \textit{filter} transformations in lines~\ref{lst:ln:logic:example:map} and~\ref{lst:ln:logic:example:filter} respectively. All other operations related to Spark are not relevant.

In this other example, the first operation detected during the analysis is the \textit{map} transformation in line~\ref{lst:ln:logic:example:map}. At this point, the analysis opts for a ``map'' strategy and prepares itself for the imminent invocation of the function passed to the \textit{map}. For convenience, all the functions in this example are depicted as lambda functions although anonymous or named classes would work as well. Once the function is invoked, the framework proceeds with the pre-processing stage, however, because no previous operations were executed, the initial input for the function is a trivial symbolic reference~($V_0$).

During the symbolic execution of the function we find that there is a branching instruction in line~\ref{lst:ln:logic:example:if}. This represents a decision point and, for this reason, a \textit{choice generator} is registered with two options: one where $V_0$ is greater than one and another where it is less than or equals to one. The control flow continues with one of the paths and stores the other for a later exploration. Given the nature of the \textit{map} transformation, the input parameter might suffer a certain transformation which, in turn, is the returned value as it is shown in line~\ref{lst:ln:logic:example:map-modifies}. During the post-processing of the operation, the symbolic expression $V_0 + 2$ is then set to be the input value of the immediate Spark operation, which in this case is a \textit{filter}. The \textit{filter} transformation is processed in a similar fashion except that in this case the input value used in its function is instantiated to whatever output generated the \textit{map} transformation. 

\begin{lstlisting}[
language=Java,
caption={[Trivial Example to Illustrate the Symbolic Execution of Spark Programs] Trivial example to illustrate the symbolic execution of spark programs with several relevant operations. The program itself has no real purpose other than to serve as a good scenario to demonstrate inter and intra procedural conditions of the analysis.},
float=t,
label=lst:logic:example
]
List<Integer> numberList = Arrays.asList(1,2,3);
JavaRDD<Integer> numbers = spark.parallelize(numberList);

numbers.map(v1 -> { /*# \label{lst:ln:logic:example:map} #*/
	if(v1 > 1) return v1; /*# \label{lst:ln:logic:example:if} #*/
	else return v1+2; /*# \label{lst:ln:logic:example:map-modifies} #*/
})
.filter(v2 -> v2 > 2); /*# \label{lst:ln:logic:example:filter} #*/
\end{lstlisting}

The function passed to the \textit{filter} transformation returns a boolean that depends on the symbolic input value. Given the nature of this kind of instruction, \spf{} registers a \textit{choice genertor} in order to explore the possible outcomes of evaluating the boolean condition. Again, one of the paths is chosen and the analysis continues. At this point there are no more relevant Spark operations and the execution comes to an end, thus, triggering a backtrack to the last unexplored path. Finally, the analysis continues until there are no more unexplored paths left.

To further illustrate the example, figure~\ref{fig:logic:symbolic-spark-example} shows the symbolic execution tree of the program. One interesting aspect to note is that the results of the \textit{map} transformation are percolated to the subsequent Spark operations; such is the case of the rightmost subtree in the symbolic execution tree. 

When observed in this way, the analysis of the program turns out to be similar to the sequential execution of the respective input functions of each of the Spark operations in the program. However, this is not always the case given that some operations have particular semantic implications, for example, \textit{flatMap} produces multiple symbolic output, hence, making it impossible to simply connect the function passed to a \textit{flatMap} as it is, to any following operation.

After the analysis is done, the module can solve the resulting path conditions and obtain a representative value in the range to produce a reduced input data set that is able to offer full path coverage of the program.


\begin{figure}[t]
	\[\xymatrix@C=0.3em @R=0.8em{ 
		& & & \ar[dd]^<(0.4){v1 = V_{0}} map & & & & \\
		& & & & & & & \\
		& & & \ar@{}[lldd]_<(0.2){true} \ar@{}[lldd]^>(0.5){V_{0}} \ar[lldd] V_{0} > 1 \ar[rrdd] \ar@{}[rrdd]_>(0.5){V_{0} + 2} \ar@{}[rrdd]^<(0.2){false} & & & & \\
		& & & & & & & \\
		& \ar[dd]^<(0.3){v2 = V_{0}} filter & & & & \ar[dd]^<(0.3){v2 = V_{0} + 2} filter & \\
		& & & & & & & \\
		& \ar@{}[ldd]_<(0.2){true} \ar[ldd] V_{0} > 2 \ar[rdd] \ar@{}[rdd]^<(0.2){false} & & & & \ar@{}[ldd]_<(0.2){true} \ar[ldd] V_{0} + 2 > 2 \ar[rdd] \ar@{}[rdd]^<(0.2){false} & \\
		& & & & & & & \\
		*+[F] \txt{$~~~V_{0} > 1$ 
			\\ $\land ~V_{0} > 2$} & & 
		*+[F] \txt{$~~~V_{0} > 1$ 
			\\ $\land ~V_{0} \leq 2$} & &
		*+[F] \txt{$~~~~~~~~~V_{0} \leq 1$ 
			\\ $\land ~V_{0} + 2 > 2$} & &
		*+[F] \txt{$~~~~~~~~~V_{0} \leq 1$ 
			\\ $\land ~V_{0} + 2 \leq 2$} \\
		V_{0} = 3 & & V_{0} = 2 & & V_{0} = 1 & & V_{0} = 0 & 
	} \]
	\caption[Symbolic Execution Tree of a Trivial Spark Program]{Symbolic execution tree corresponding to the Spark program shown in listing~\ref{lst:logic:example}. The input set $\{3,2,1,0\}$ represents a minimal input set that would explore all feasible paths in the program.}
	\label{fig:logic:symbolic-spark-example}
\end{figure}

\TUsection{JPF-SymSpark}
\label{sec:jpf-symspark}

% Spark mocked library
% Instruction Factory
% Listener and Coordinator
% Strategies
JPF-SymSpark is a \jpf{} module whose goal is to coordinate the symbolic execution of Apache Spark programs to produce a reduced input dataset that ensures full path coverage on a regular execution. It builds on top of \spf{} to delegate the handling of symbolic expressions while it focus on how to interconnect Spark's transformations and actions in order to reason coherently over the execution flow of the program. This section describes the general structure and technical aspects of the \textit{JPF-SymSpark} module. The work presented here is based on the logical processes defined in the previous section. 

The main \jpf{} extension points used in the module are:

\begin{itemize}
	\item \textbf{Bytecode Factory}: Implemented in the \textit{SparkSymbolicInstructionFactory} class, this bytecode instruction factory is in charge of detecting relevant Spark instructions.
	\item \textbf{Listeners}: The main listener is the \textit{SparkMethodListener} class. After the bytecode instruction factory, the listener provides the main interaction point with the analyzed program. In our case, the listener aims to orchestrate the correct symbolic execution of a sequence of Spark operations.
	\item \textbf{Publishers}: The \textit{SparkMethodListener} class also works as a publisher (these two responsibilities are often held together by the same class). The class produces a reduced input dataset obtained after selecting concrete values that satisfy the path conditions.
	\item \textbf{Choice Generators}: In some cases, additional choice generators must be registered in order to correctly reproduce the behavior of some of the Spark operations. Such is the case of the \textit{SparkMultipleOutputChoiceGenerator} class, which is used when analyzing a \textit{flatMap} transformation, given that it can potentially produce multiple outputs.
\end{itemize}

Additionally, the module also introduces the following control components that play an important role during the analysis:

\begin{itemize}
	\item \textbf{Validators}: This are abstractions that encapsulate the necessary validations to determine if an executed instruction is a Spark operation or an invocation of the user-defined function. As for the case of the Java platform, the validator is implemented in the \textit{JavaSparkValidator} class.
	\item \textbf{Method Strategies}: Each Spark operation follows a different strategy depending on how they deal with their input and output parameters. One example of a method strategy can be found in the \textit{FilterStrategy} class.
	\item \textbf{Method Coordinator}: The coordinator is in charge of selecting the adequate strategy based on the Spark operation currently being executed. It maintains a general view of the whole analysis. The method coordinator is implemented in the \textit{MethodSequenceCoordinator} class.
	\item \textbf{Surrogate Spark Library}: The regular Spark library undergoes several unnecessary workload that is not relevant to the symbolic execution. Instead of it, a surrogate Spark library was implemented and included as one of the dependencies of the module to relieve the analysis of the irrelevant load.
\end{itemize}

The following sections explain in more detail the different components that conform the module and what role do they play in the whole analysis.

%TODO: Maybe mention that the module was created using the jpf-template project
%TODO: Mention the structure and key aspects, for example the configuration properties, and the new Instruction Factory

\TUsubsection{Spark Library and JPF}

When executing an analysis using \jpf{}, the whole program is run under an instrumented \acrshort{acr:jvm} that keeps track of the execution state of the program. \jpf{} considers every program statement executed, even if it is executed by third-party libraries or dependencies indirectly invoked in the system under test. These libraries must be included in the \jpf{}'s classpath (which is a different classpath than the normal Java classpath of the system under test) if they are executed, because if not, \jpf{} will fail indicating that it is not able to find certain references during execution.

On the other hand, Spark, as many other modern applications, depends on a constantly growing number of external libraries. To execute an analysis on a Spark program with \jpf{} one could include all this libraries and dependencies in the the \jpf{}'s classpath and let tool handle all the invocations internally. However, this approach has several problems: First, the execution of more statements increases the workload and state space of \jpf{}. Second, some of Spark's operations handle native calls that, for example, deal with the way tasks are placed in the OS; \jpf{} does not handle such native operations by default, which leads to the need of creating surrogate peers that mock the behavior of such calls. Lastly and more importantly, most of these operations are called in methods that are unrelated to the actions and transformations that are relevant to the symbolic execution, leading to an unnecessary overhead that does not provide any benefit.

Because of all these reasons, including the Spark library and all its dependencies was not a reasonable approach. Instead, we decided mock up the Spark library, in order to mimic some of the classes that participate in a Spark program. The idea is to minimize the number of external dependencies and native calls while at the same time replacing the implementations of methods irrelevant to the analyses with simplified versions of themselves. Listing~\ref{lst:module:java-spark-context} is an example of how a class that is irrelevant to the analysis is simplified. In the regular Spark library the \textit{JavaSparkContext} class triggers a lot of heavy processes, like initializing the whole Spark framework; now it is just reduced to empty or simple code blocks.

%TODO: Include lines references in description
\begin{lstlisting}[
language=Java,
caption={[Mocked JavaSparkContext] Mocked version of the \textit{JavaSparkContext} class. The methods are as simple as they could be while still maintaining the contract of the original class. Note that the classes \textit{SparkConf} and \textit{JavaRDD} are also mocked.},
float=h,
label=lst:module:java-spark-context
]
import java.util.Arrays;
import java.util.List;
import org.apache.spark.SparkConf;

public class JavaSparkContext {	
	public JavaSparkContext(SparkConf conf){}
	public void stop() {}	
	public void close() {}
	public <T> JavaRDD<T> parallelize(List<T> list) {		
		return new JavaRDD<T>(list);
	}
	public JavaRDD<String> textFile(String file) {
		return new JavaRDD<String>(Arrays.asList(""));
	}
}
\end{lstlisting}

However, some of the methods invoked by the Spark library are relevant to the analysis. Such is the case of the methods defined in the \textit{JavaRDD} class and the rest of the classes in the \acrshort{acr:rdd} family. These methods include operations like \textit{filter}, \textit{map} and \textit{reduce}, that make use of the functions passed by the programmers. In these cases, it is extremely relevant that the passed function is invoked inside these methods so the analysis can be triggered following the usual \spf{} approach. Listing~\ref{lst:module:filter-javardd} shows an example of the mocked \textit{filter} method of the \textit{JavaRDD} class. The function passed to the \textit{filter} method is invoked with the first element of the \acrshort{acr:rdd} only and the returned value is the current RDD itself given that it does not affect the end result when using symbolic input parameters.

%TODO: Include lines references in description
\begin{lstlisting}[
language=Java,
caption={[Mocked \textit{filter} method in the JavaRDD class] Mocked filter method in the JavaRDD class. The function passed to the method is invoked. Note that the \textit{Function} interface is also mocked.},
float=h,
label=lst:module:filter-javardd
]
public JavaRDD<T> filter(Function<T,Boolean> f) {		
	try {
		f.call(list_t.get(0));
	} catch (Exception e) {
		e.printStackTrace();
	}
	return this;
}
\end{lstlisting}

The surrogate Spark library is already included into the dependencies of the \textit{JPF-SymSpark} module. Nevertheless, the implementation is not extensive, which might require further expansion as the different cases and programs require. Moreover, the library is bound to version 2.0.2 of the original Spark library, which poses a drawback in terms of consistency if the core behavior of Spark changes in future versions. More about this drawback can be found in section~\ref{sec:limitations}.

Having effectively discarded irrelevant portions of the system under test by the means of the mocked up library, it is simpler to identify the relevant Spark operations that have an impact on the analysis.
\TUsubsection{Instruction Factory}

% Point out how the detection of Spark operations as well relies on the combinations of the Instruction Factory, the instruction class itself and Platform-specific validator classes.

% Indicate that the only relevant instruction is INVOKEVIRTUAL

% Show a comparison between how a code looks and how its compiled bytecode instruction looks

The first step for carrying out the analysis is to identify when a Spark operation is being executed. Given that all relevant operations in the concrete \textit{JavaRDD} class are implemented as non-static methods, the bytecode instruction of interest is \textit{invokevirtual}. This instruction is in charge of dispatching Java methods, unless they are interface methods, static methods or some other special cases (\textit{invokeinterface}, \textit{invokestatic} and \textit{invokespecial} are used respectively)~\cite{Lindholm2014}.

For this purpose, we implemented the \textit{SparkSymbolicInstructionFactory} class, which extends from the \textit{SymbolicInstructionFactory} class defined in the \spf{} module. The goal of this class is to solely intercept calls to the \textit{invokevirtual} bytecode instruction and validate if they intend to dispatch one of the Spark operations relevant to the analysis.

Just to illustrate this situation better in the case of the Java implementation, let us assume that the \textit{filter} transformation is being called on an existing \acrshort{acr:rdd} such as

\hspace*{1cm} \lstinline[language=Java,]|rdd.filter(...)|

then, the corresponding bytecode will look like the following

\hspace*{1cm} \lstinline[]|invokevirtual PATH/JavaRDD.filter:(PATH/Function;) PATH/JavaRDD;|

with ``PATH'' representing the full package path where the classes or interfaces are located. The rest of the instruction represents the method name and the method descriptor; sufficient information for identifying the relevant operations. The function parameter was intentionally omitted because, although the passed parameter must implement the \textit{Function} interface, this can be done in several ways; being the two most common ways lambda expressions and anonymous classes. At this point, neither of this two approaches represent a difference when detecting the Spark operation, however, it will require special attention later on when detecting the invocation of the passed function.

The instruction factory that we implemented, makes use of a validator, which is in charge of detecting the concrete method and class names that are particular for each Spark implementation. At the moment we only support the Java implementation, although the framework is flexible to support additional validators, for example, one that could detect the Scala implementation of Spark.

Moreover, the user must define in a configuration file the method names of the operations of interest (e.g., map, filter) in order to indicate this type of methods should be analyzed. The method names must be defined in a \texttt{.jpf} file, in a similar fashion as \spf{}, under the key \texttt{spark.methods}; in the case of having multiple methods of interest, they must be separated with a semicolon.

Once the method has been detected, the instruction factory issues a custom \texttt{INVOKEVIRTUAL} instruction to the \jpf{} core. This instruction indicates what should the \jpf{} virtual machine do when the method of interest is executed. In our case, we need to prepare for the subsequent execution of the function passed as a parameter to the Spark operation. To do this, we manipulate the \jpf{} configuration programatically in order to take advantage of the configuration properties used by \spf{} to define which methods are supposed to be analyzed by the symbolic engine.

Again, let us assume this time that the \textit{filter} method above was executed in a class named \texttt{Main} (fully described in \texttt{com.test.Main}). Then, after detecting the \textit{filter} transformation, the configuration will contain one of the following entries in the \texttt{symbolic.method} key:

\hspace*{1cm} \lstinline[]|symbolic.method=...;com.test.Main$1.call(sym)|  \hfill (1)

or

\hspace*{1cm} \lstinline[]|symbolic.method=...;com.test.Main.lambda$main$0(sym)| (java compiler) \\
\hspace*{1cm} \lstinline[]|symbolic.method=...;com.test.Main.lambda$0(sym)| (eclipse compiler) \hfill (2)

depending if the passed function was implemented as an anonymous class or a lambda expression. 

The ``\texttt{\$1}'' in (1) points to the first anonymous class created inside \texttt{com.test.Main} (the qualified path for an anonymous class is always separated by a ``\texttt{\$}'' sign). The number is monotonically increasing according to how many anonymous classes have been created up to that point. It is important to note that the method \textit{call} indicated here corresponds to the implementation of the method \textit{call} defined in the \textit{Function} interface.

In the case of (2), the method indicated corresponds to a static method defined by the Java compiler whenever a lambda expression is found. This method will invoke afterwards an anonymous class created on the fly which implements the \textit{call} method of the \textit{Function} interface. It is sufficient to indicate the first static method with the symbolic parameter given that it only forwards the execution to the \textit{call} method in the anonymous class. This solution came as a workaround for detecting lambda expressions, given that \spf{} does not provide any mechanism on its own to clearly specify the analysis of such methods. Same as in the case of anonymous classes, the number accompanying the method is monotonically increasing and depends on how many lambda expressions have been created thus far.

It will be relevant later on, that the method marked to be symbolically executed by (1) will be invoked using \texttt{invokevirtual}, while the ones defined by (2) will be invoked using \texttt{invokestatic}. This will require the handling of the input and output parameters in a different way.

The dynamic manipulation of the configuration properties makes it easier for the user to specify which methods are to be symbolically executed. Defining the methods before hand, as in the regular \spf{} approach, proves to be cumbersome, given that the method names corresponding to anonymous classes or lambda expressions are often elusive and even difficult to track.

For further information on how anonymous functions and lambda expressions are compiled and represented in Java bytecode please refer to the Java Language Specification~\cite{Gosling2014}.

%TODO: Include the detection of lambda expressions as a collaboration?


\TUsubsection{Spark Listener and Method Sequence Coordinator}

The overall process of the \textit{JPF-SymSpark} module is implemented by the \textit{SparkMethodListener} and the \textit{MethodSequenceCoordinator}. The listener acts as a stateless interpreter of the analysis' progress while the coordinator behaves as a stateful control node that provides coherence to the whole process. Both entities work together closely, being the listener just an entry point that dispatches actions to the coordinator based on a selected few relevant events.

\TUsubsubsection{Spark Method Listener}

The Spark method listener extends the \textit{PropertyListenerAdapter} which already provides entry points to all the events exposed by \jpf{}. The relevant events in our case are:

\begin{itemize}
	\item \textbf{instructionExecuted}: Which is triggered every time an instruction is executed. When this event is triggered, the listener forwards the executed instruction to the coordinator in order to validate if it is a relevant instruction for the analysis.
	\item \textbf{methodExited}: Which is triggered every time a method finishes its execution. In this case, if the method is related to Spark, the listener indicates to the coordinator that it should prepare to percolate the output to possible subsequent Spark methods.
	\item \textbf{stateAdvanced} Which is triggered every time the internal state of \jpf{} advances. This is only relevant to identify that an end state has been reached, in which case the listener indicates to the coordinator that a full path has been explored.
	\item \textbf{stateBacktracked}: Which is triggered every time the execution reached an end state but there were already choice generators registered with unexplored options. The state is always backtracked to the latest point where a choice generator was registered. In this case, the listener instructs the coordinator to obtain a solution to the symbolic input value if the state backtracked to a different alternative of the path condition.
\end{itemize}

Additionally to these events, the \textit{SparkMethodListener} also implements the \textit{PublisherExtension} interface which allows it to act as a publisher as well. The combination of a listener and a publisher is a common practice among the \jpf{} modules, given it is convenient to collect information worth to be published during the different events tracked by the listener. As a publisher, when the whole analysis is done, the different sample values that satisfied the path conditions that were collected by the coordinator are displayed in the console as a single input dataset. Section~\ref{subsec:module:output} explains in more detail how the output is generated.

\TUsubsubsection{Method Sequence Coordinator}

The \textit{MethodSequenceCoordinator} class was created as a stateful control structure used to keep track of the progress and results of the analysis. Although it is heavily influenced by the events detected in the listener, the idea was to keep it detached from the event-flow responsibilities of the listener, focusing only on the module's process state.

The goal of the coordinator is to switch to the adequate strategy based on the Spark operations. The different strategies actually do the heavy-lifting in the analysis, and are in charge of handling the behavior of each particular operation correctly. More on the strategies can be found in section~\ref{subsec:module:strategies}.

The main actions of the coordinator are:

\begin{itemize}
	\item \textbf{detectSparkInstruction}: This action determines whether the executed instruction is a Spark operation or the respective invocation if its parameter function. In the case of being a Spark operation, the coordinator selects the adequate strategy matching the operation. Otherwise, if it detects the invocation of the parameter function, then it indicates to the currently active strategy to execute the pre-processing phase.
	\item \textbf{percolateToNextMethod}: In this case, the coordinator instructs the active strategy to execute its post-processing phase. In general, all post-processing activities deal with the inter-connection of Spark operations, this is, dealing with output parameters and the interruption of the control flow, among others.
	\item \textbf{processSolution}: Arguably the most important action of the coordinator. This action is invoked anytime \jpf{} backtracks to a previous state. In the case the state is backtracked to a point where a different path must be taken in a branching condition, that means that the other path has been completely explored. Being this the case, the current path condition is obtained from the choice generator and passed to the solver to determine if it is feasible or not; in the positive case, a sample value of the symbolic input is added to the solution list, otherwise an unfeasible path is reported.
\end{itemize}

% Define block styles
\tikzstyle{block} = [rectangle, draw, fill=white, align=center,rounded corners, minimum height=3em, minimum width=3em]
\tikzstyle{invisible-block} = [rectangle, draw=none, fill=none, minimum width=3cm, minimum height=4mm, node distance=4.9mm]
\tikzstyle{line} = [draw, -{>[length=2mm, width=1mm]}]
\tikzstyle{line-dashed} = [draw, dotted, -{>[length=2mm, width=1mm]}]
\begin{figure}[t]
	\centering
	\begin{tikzpicture}[node distance=2cm, auto]
	% Place nodes
	\node [block, draw=none, align=left] (CODE) {
		\\
		\texttt{...} \\
		\texttt{numbers.map(} \\
		\texttt{~~v1 -> \{} \\
		\texttt{~~~~if(v1 > 1) return v1;} \\
		\texttt{~~~~else return v1+2;} \\
		\texttt{~~\}} \\
		\texttt{)} \\
		\texttt{.filter(v2 -> v2 > 2);} \\
		\texttt{...} \\		
	};
	
	\node [invisible-block, minimum height=5cm, minimum width=3cm] (INVISIBLE){};
	\node [invisible-block, below of=INVISIBLE] at (INVISIBLE.north) (C1){};
	\node [invisible-block, below of=C1, minimum width=0.7cm](C2){};
	\node [invisible-block, below of=C2, minimum width=0.7cm](C3){};
	\node [invisible-block, below of=C3](C4){};
	\node [invisible-block, below of=C4](C5){};
	\node [invisible-block, below of=C5, minimum width=0.7cm](C6){};
	\node [invisible-block, below of=C6, minimum width=2cm](C7){};
	\node [invisible-block, below of=C7, xshift=0.6cm](C8){};
	\node [invisible-block, below of=C8, xshift=-0.6cm, minimum width=2cm](C9){};
	
	\node [block, right of=CODE,  drop shadow, node distance=6.5cm, minimum height=5cm, minimum width=3cm] (LISTENER) {Spark \\ Method \\ Listener};
	\node [invisible-block, below of=LISTENER] at (LISTENER.north) (L1){};
	\node [invisible-block, below of=L1](L2){};
	\node [invisible-block, below of=L2](L3){};
	\node [invisible-block, below of=L3](L4){};
	\node [invisible-block, below of=L4](L5){};
	\node [invisible-block, below of=L5](L6){};
	\node [invisible-block, below of=L6](L7){};
	\node [invisible-block, below of=L7](L8){};
	\node [invisible-block, below of=L8](L9){};
	
	\node [block, right of=LISTENER,  drop shadow, node distance=6.5cm, minimum height=5cm, minimum width=3cm] (COORDINATOR) {Method \\ Sequence \\ Coordinator};
	\node [invisible-block, below of=COORDINATOR] at (COORDINATOR.north) (CO1){};
	\node [invisible-block, below of=CO1](CO2){};
	\node [invisible-block, below of=CO2](CO3){};
	\node [invisible-block, below of=CO3](CO4){};
	\node [invisible-block, below of=CO4](CO5){};
	\node [invisible-block, below of=CO5](CO6){};
	\node [invisible-block, below of=CO6](CO7){};
	\node [invisible-block, below of=CO7](CO8){};
	\node [invisible-block, below of=CO8](CO9){};
			
	% Draw edges
	\path [line-dashed] (C2.east) -- node[right=0.85cm, above=0.5mm]{\footnotesize \textit{instructionExecuted}} (L2.west);
	\path [line-dashed] (C3.east) -- node[right=0.85cm, above=0.5mm]{\footnotesize \textit{instructionExecuted}} (L3.west);
	\path [line-dashed] (C7.west) -- node[right=1.9cm, above=0.5mm]{\footnotesize \textit{methodExited}} (L7.west);
	\path [line-dashed] (C8.east) -- node[]{\footnotesize \textit{instructionExecuted}} (L8.west);
	\path [line-dashed] (C9.west) -- node[right=1.71cm, above=0.5mm]{\footnotesize \textit{stateBacktracked}} (L9.west);
	
	\path [line] (L2.east) -- node[]{\footnotesize \textit{detectSparkInstruction}} (CO2.west);
	\path [line] (L3.east) -- node[]{\footnotesize \textit{detectSparkInstruction}} (CO3.west);
	\path [line] (L7.east) -- node[]{\footnotesize \textit{percolateToNextMethod}} (CO7.west);
	\path [line] (L8.east) -- node[]{\footnotesize \textit{detectSparkInstruction}}(CO8.west);
	\path [line] (L9.east) -- node[]{\footnotesize \textit{processSolution}}(CO9.west);
	\end{tikzpicture}
	\caption[Execution flow of the Spark Method Listener and Coordinator]{Execution flow of the Spark method listener and the coordinator when analyzing the sample program shown in listing~\ref{lst:logic:example}. The exact point where the listener events are triggered actually depends on precise bytecode instructions; this diagram provides an approximation to where this instructions occur in the source code.}
	\label{fig:listener:flow-diagram}
\end{figure}

The diagram depicted in~\ref{fig:listener:flow-diagram} shows at which point during the execution of the program under test will the listener events be triggered. The \textit{instructionExecuted} event gets triggered on every instruction, however, the depicted points only refer to those instructions that are relevant to the analysis and will cause a respective action in the coordinator; the namely instructions correspond to the invocation of a Spark operation or its respective parameter function. Likewise, the \textit{methodExecuted} event is only relevant when the Spark operation or the parameter function finish. The \textit{stateBacktracked} is indicated to occur at a later point, usually at the end of the program although the backtrack process can be forced without necessarily having reached this point.
\TUsubsection{Method Strategies}
\label{subsec:module:strategies}

The method strategies specify the concrete behavior of the analysis for each relevant Spark operation. They implement how the analysis is to be carried out, particularly during the pre-processing and post-processing phases. Additionally, they maintain a reference to the input and output values of the operation.

The supported Spark operations are: \textit{filter}, \textit{map}, \textit{reduce} and \textit{flatMap}.

This section explains what particular conditions apply to each strategy and how are they carried out through the analysis.

\TUsubsubsection{\textit{filter}}

The purpose of the \textit{filter} transformation is to produce a new \acrshort{acr:rdd} containing only the elements that satisfy a given predicate. The predicate is passed to the transformation in the form of a boolean function that is invoked for each element of the \acrshort{acr:rdd}. Because of this reason, the \textit{filter} transformation itself always implies at least two possible execution paths, one for those who satisfy the predicate and one for those who do not. Figure~\ref{fig:strategies:filter} depicts the symbolic execution of a \textit{filter} transformation according to the filter strategy.

% Define block styles
\tikzstyle{block} = [rectangle, draw, fill=white, align=center,rounded corners, minimum height=3em, minimum width=3em, drop shadow]
\tikzstyle{invisible-block} = [rectangle, draw=none, fill=none, align=center, minimum height=3em, minimum width=3em]
\tikzstyle{triangle} = [isosceles triangle, draw, fill=white, align=center, shape border rotate=-180, minimum height=3em, minimum width=2cm, drop shadow]
\tikzstyle{line} = [draw, -{>[length=2mm, width=1mm]}]
\begin{figure}[h]
	\centering
	\begin{tikzpicture}[node distance=6cm, auto]
	% Place nodes	
	\node [invisible-block] (INIT) {...};
	\node [triangle, right of=INIT] (FILTER) {\textit{filter}};
	\node [invisible-block, right of=FILTER, minimum width=2cm] at (FILTER.right corner) (TRUE) {...};
	\node [block, right of=FILTER] at (FILTER.left corner) (BACKTRACK) {Backtrack};
	
	% Draw edges
	\path [line] (INIT) -- node[]{\footnotesize $SYM\_EXPR$}(FILTER.apex);
	\path [line] (FILTER.right corner) -- node[below]{\textit{true}} node[]{\footnotesize $SYM\_EXPR$}(TRUE);
	\path [line] (FILTER.left corner) -- node[below]{\textit{false}}(BACKTRACK);
	
	\end{tikzpicture}
	\caption[Diagram of the \textit{filter} Strategy]{Diagram of the \textit{filter} strategy. The input parameter is initialized to the symbolic expression percolated by the previous Spark operation if there is any, otherwise, a new symbolic expression is used instead. Filter transformations always produce at least two branches, one that satisfies the predicate and one that does not. In the case of the branch that does not satisfy it, the execution is terminated and an immediate backtrack is triggered. The output of the transformation is always the same input symbolic expression disregarding the path taken.}
	\label{fig:strategies:filter}
\end{figure}

\textit{Pre-processing}

During the pre-processing phase, the filter strategy only checks for the invocation of the parameter function and validates if it is coming from an \texttt{invokevirtual} or an \texttt{invokestatic} bytecode instruction. If it is coming from an \texttt{invokevirtual} (the function was implemented as an anonymous class), the strategy manipulates the stack frame of the current invocation and replaces the element in the second position with the symbolic expression passed by the coordinator. This element corresponds to the input parameter of the passed function (i.e., an element of the \acrshort{acr:rdd}). The replacement of the second element is necessary because, if this is not done, \spf{} will call the function with a new symbolic expression instead, breaking the continuity of the analysis. The reason the second element is the one replaced instead of the first is that, in the stack frame of an \texttt{invokevirtual} instruction, the first position contains a reference to the invoking object instead (i.e., a reference to \texttt{this}). On the contrary, if the invocation of the function comes from a \texttt{invokestatic} instruction (the function was implemented as a lambda expression), then the first element is replaced in the stack frame because \texttt{invokestatic} instructions do not have references to the invoking object itself.

Given that the \textit{filter} transformation always implies a fork in the control flow, it is most likely that a choice generator was registered by \spf{}
during the execution of the method in order to explore the possible paths. Needless to mention, this only occurs if the symbolic expression passed to the function participates in any of the conditional operations, otherwise, no choice generator is registered; then again, this scenario is irrelevant given that it means that the filter does not act upon the values of the \acrshort{acr:rdd}.

\textit{Post-processing}

On the post-processing phase, the strategy checks if the exited method is actually the \textit{filter} transformation and proceeds to obtain the last registered choice generator. As explained above, this choice generator will always be a path condition choice generator registered by \spf{}. Then the strategy proceeds to validate if the path currently executing corresponds to the negative evaluation of the predicate, in which case the execution of the thread is abruptly interrupted and a backtrack action is forced. The reason for this relies upon the fact that exploring a path with a negative filter condition is not relevant given that this path will never be executed in a Spark program. However, by forcing the backtrack action, the analysis is guided to find a solution that satisfies that path, which ends in a value that does not pass the filter (necessary for full path coverage).

The symbolic input of a \textit{filter} transformation does not suffer any permanent modification during its execution. For this reason, the output of the \textit{filter} transformation is the same input value passed to the function during the pre-processing phase.

\TUsubsubsection{\textit{map}}

The \textit{map} transformation constructs a new \acrshort{acr:rdd} containing the result of applying the parameter function to each element of the initial \acrshort{acr:rdd}. Normally, the input value passed to the parameter function is used in the operation that produces the output value, hence making the output be a derivation of the input value. Considering this rationale in the context of a symbolic execution, the output of the parameter function passed to a \textit{map} transformation is defined in terms of the input symbolic expression. Figure~\ref{fig:strategies:map} depicts the symbolic execution of a \textit{map} transformation according to the map strategy.

% Define block styles
\tikzstyle{block} = [rectangle, draw, fill=white, align=center,rounded corners, minimum height=3em, minimum width=3em, drop shadow]
\tikzstyle{invisible-block} = [rectangle, draw=none, fill=none, align=center, minimum height=3em, minimum width=3em]
\tikzstyle{triangle} = [isosceles triangle, draw, fill=white, align=center, shape border rotate=-180, minimum height=3em, minimum width=2cm, drop shadow]
\tikzstyle{line} = [draw, -{>[length=2mm, width=1mm]}]
\begin{figure}[h]
	\centering
	\begin{tikzpicture}[node distance=6cm, auto]
	% Place nodes	
	\node [invisible-block] (INIT) {...};
	\node [triangle, right of=INIT] (MAP) {\textit{map}};
	\node [invisible-block, right of=MAP] at (MAP.right corner) (SCN1) {...};
	\node [invisible-block, right of=MAP] at (MAP.30) (SCN2) {...};
	\node [invisible-block, right of=MAP] at (MAP.left corner) (SCN3) {...};
	
	% Draw edges
	\path [line] (INIT) -- node[]{\footnotesize $SYM\_EXPR$}(MAP.apex);
	\path [line] (MAP.right corner) -- node[]{\footnotesize $SYM\_EXPR^I$}(SCN1);
	\path [line] (MAP.30) -- node[below=3mm]{...} node[]{\footnotesize $SYM\_EXPR^{II}$}(SCN2);	
	\path [line] (MAP.left corner) -- node[]{\footnotesize $SYM\_EXPR^N$}(SCN3);
	
	\end{tikzpicture}
	\caption[Diagram of the \textit{map} Strategy]{Diagram of the \textit{map} strategy. The input parameter is initialized to the symbolic expression percolated by the previous Spark operation if there is any, otherwise, a new symbolic expression is used instead. The output value percolated to the following functions is a new symbolic expression derived from any operations applied on the symbolic input. Map transformations do not necessarily have a branching condition, however, in the case there are, it is most likely that the output value will be different.}
	\label{fig:strategies:map}
\end{figure}

\textit{Pre-processing}

The pre-processing phase is identical to the pre-processing phase of the filter strategy. The handling of the input parameters of the passed function is carried out in the same way given that, in both cases, the passed functions have the same number of input parameters, hence their stack frame behaves the same.

\textit{Post-processing}

In the post-processing phase, the map strategy waits until the passed function finishes its execution. This is done by checking if the exited method matches the full descriptor of the \textit{call} method in either the anonymous class or lambda expression that represents the passed function. Once it detects the right exited method then it stores its output value in an attribute of the strategy. This value will be used the next time strategies are switched and it will be used as the input value of the next Spark operation.

A \textit{map} transformation does not have a branching condition necessarily; it might be the case that the \acrshort{acr:rdd} is just manipulated and an output value returned. However, even in this case, the transformation of the symbolic input needs to be tracked in order to percolate it to the next Spark operation. Chaining output and input values between Spark operations is of utmost importance in order to build precise path conditions that faithfully represent the control flow of the program.

In the case the \textit{map} transformation incurs in any branching condition, then \spf{} will register the respective choice generators and all the options will be explored accordingly. Such a case leads to potentially different outputs depending on which path was taken.


\TUsubsubsection{\textit{reduce}}

The \textit{reduce} action produces a single output value resulting from the combination of all the elements in the \acrshort{acr:rdd}. The combination is defined in the function passed to action, which has to fulfill the properties of being commutative and associative. The function passed to the action implements the \textit{Function2} interface, whose main difference is that it takes two input parameters: the first is the accumulated value of the operation so far, and the second represents one element in the \acrshort{acr:rdd}. The behavior of a \textit{reduce} action resembles a full scan of the elements of the \acrshort{acr:rdd}, carrying always the accumulated value iteration over iteration.

Reduce actions can be analyzed following two different strategies: one where the accumulated parameter of the function is not considered as a symbolic variable and another where it is. The strategy discussed next refers to the former; the latter is explained in the subsequent section. The user indicates which mode to use by specifying it in the configuration file.

\textit{Pre-processing}

This strategy considers the input parameter corresponding to a single element in the \acrshort{acr:rdd} as the percolated symbolic expression, however, the parameter corresponding to the accumulated value of the reduce action is considered as a concrete input. 

The analysis takes advantage of the concolic operational mode of \spf{} by defining the method to be inspected with the first parameter as a concrete input (achieved by using the ``con'' keyword instead of ``sym'' in the fully qualified method name). Afterwards, in a similar fashion as in the filter and map strategies, the second or third element in the stack frame is replaced with the percolated symbolic expression (depending if the instruction is \texttt{invokevirtual} or \texttt{invokestatic}). Figure~\ref{fig:strategies:reduce:concrete} illustrates this mode.

\textit{Post-processing}

Independently of the output of the \textit{reduce} action, the post-processing phase of the reduce strategy always triggers a termination of the current execution thread and a backtrack. This is because of the nature of Spark actions which indicate the culmination of the processing of an \acrshort{acr:rdd}. However, during the analysis of the reduce action, there might be branching operations executed on the single parameter, which will cause \spf{} to register a choice generator and explore all the possible paths. Ultimately, all these paths will lead to a backtrack after the end of the \textit{reduce} action.

\TUsubsubsection{\textit{iterativeReduce}}

The second strategy that can be used when analyzing \textit{reduce} actions considers both parameters of the passed function as symbolic. The direct consequence of this consideration is that the symbolic engine has to reason now over a variable that represents an accumulated value resulting from the application of the passed function several times. Given that the number of elements in the \acrshort{acr:rdd} is irrelevant for the sake of the analysis, the user has to specify how many iterations of the \textit{reduce} action will be carried out in order to ensure termination. However, path explosion can occur easily given that each iteration makes the number of reachable states grow exponentially.

The complexity of this strategy is noticeable in comparison to the other strategies. The expected output dataset now becomes a family of datasets corresponding to the number of elements indicated to be in the \acrshort{acr:rdd}. The path conditions take into consideration multiple symbolic variables that have to comply with all the transformations and constraints collected so far. The particular aspects of the output datasets are detailed in Section~\ref{subsec:module:output}.

% Define block styles
\tikzstyle{block} = [rectangle, draw, fill=white, align=center,rounded corners, minimum height=3em, minimum width=3em, drop shadow]
\tikzstyle{invisible-block} = [rectangle, draw=none, fill=none, align=center, minimum height=3em, minimum width=3em]
\tikzstyle{diamond} = [shape=diamond, draw, fill=white, align=center, minimum height=3em, minimum width=2cm, drop shadow]
\tikzstyle{line} = [draw, -{>[length=2mm, width=1mm]}]

\begin{figure}[t]
	\centering
	\begin{tikzpicture}[node distance = 4cm, auto]	
	% Place nodes	
	\node [invisible-block] (INIT) {...};
	\node [diamond, right of=INIT, node distance=8cm] (REDUCE) {\textit{reduce}};
	\node [block, right of=REDUCE] at (REDUCE.east) (BACKTRACK) {Backtrack};
	
	% Draw edges
	\path [line] (INIT) -- node[]{\footnotesize $(CONC, SYM\_EXPR)$}(REDUCE.west);
	\path [line] (REDUCE.east) -- (BACKTRACK);	
	\end{tikzpicture}
	
	\begin{minipage}[t]{\linewidth}
		\subcaption{Reduce strategy with the accumulated value taken as a concrete input.} 
		\label{fig:strategies:reduce:concrete}
	\end{minipage}
	\vspace{\belowdisplayskip}
	
	\begin{tikzpicture}[node distance = 4cm, auto]	
	% Place nodes	
	\node [invisible-block] (INIT) {...};
	\node [diamond, right of=INIT, node distance=8cm] (REDUCE) {\textit{reduce}};
	\node [block, right of=REDUCE] at (REDUCE.east) (BACKTRACK) {Backtrack};
	
	% Draw edges
	\path [line] (INIT) -- node[]{\footnotesize $(SYM\_EXPR^I, SYM\_EXPR^{II})$}(REDUCE.west);
	\path [line] (REDUCE.east) -- (BACKTRACK);
	\path [line] (REDUCE) edge [loop above] node[]{\footnotesize $(SYM\_EXPR^{X_n}, SYM\_EXPR^{Y_n})$} ();	
	\end{tikzpicture}
	
	\begin{minipage}[t]{\linewidth}
		\subcaption{Reduce strategy with the accumulated value taken as a symbolic input. The value $n$ represents \\ a given iteration.}
		\label{fig:strategies:reduce:symbolic}
	\end{minipage}
	
	\caption[Diagrams of the \textit{reduce} Strategy]{Diagrams of the \textit{reduce} Strategy. The two modes of execution are shown here. The first considers the accumulated input parameter as concrete, this translates to only considering branching operations applied on single elements of the \acrshort{acr:rdd}. The other mode considers the accumulated to be a symbolic input as well and iterates over the operation a fixed number of times. A backtrack is always triggered after a reduce action.}
\end{figure}

\textit{Pre-processing}

As mentioned before, this strategy considers both parameters of the function passed to the \textit{reduce} action as symbolic variables. Once the \textit{reduce} actions is detected, a new choice generator called \textit{SparkIterativeChoiceGenerator} is registered. This choice generator is used to keep a counter with the number of times the function has to be executed and it also keeps track of the output of each of the iterations; it is not used as a value provider in a direct sense rather as a placeholder to keep track of the iterative execution. The current path condition, if any, is also kept in the choice generator along with the single symbolic variable used so far. This value will serve as a template for creating new symbolic variables that comply with the conditions accumulated to this point.

Once the execution of the \textit{call} method is detected, the strategy follows one of two approaches: If there are no accumulated values registered in the \textit{SparkIterativeChoiceGenerator} choice generator then a new symbolic expression is created taking into consideration the base structure of the percolated expression. This new expression is used as the first accumulated value while the percolated expression is used as the single input value. On the other case, an output value (representing the result of a previous iteration) is taken from the choice generator and set as the accumulated value while a new symbolic expression is produced and set as the regular single input value of the function. Figure~\ref{fig:strategies:reduce:symbolic} illustrates this mode.

As a technical note, symbolic expressions and path conditions are implemented in \spf{} following the \textit{visitor} design pattern~\cite{Gamma1994}. New visitors were implemented in order to create copies of existing expressions and conditions among others.

\textit{Post-processing}

Once the \textit{call} method finishes execution the current path condition is extended by enforcing the same initial constraints on the newly generated expressions for this iteration. Additionally, any constraints accumulated from a previous iteration must also be included in the current path condition. This has to be done at this point because \jpf{} is not really executing an iteration, instead the iteration is simulated with the \textit{SparkIterativeChoiceGenerator}, which will cause the path conditions to be solved after the method finishes and the engine is set to backtrack to the latest point (most likely the point where the \textit{SparkIterativeChoiceGenerator} was registered), hence missing interconnection between the iterations. Including this conditions ensure that the right path in the symbolic execution tree is taken.

Lastly, the output expression of the method is stored in the choice generator along with the current path condition. This will serve as an input for a subsequent iteration in case the maximum number of iterations has not been reached yet.

\TUsubsubsection{\textit{flatMap}}

% Define block styles
\tikzstyle{block} = [rectangle, draw, fill=white, align=center, minimum height=4em, minimum width=3em, drop shadow]
\tikzstyle{invisible-block} = [rectangle, draw=none, fill=none, align=center, minimum height=3em, minimum width=3em]
\tikzstyle{triangle} = [isosceles triangle, draw, fill=white, align=center, shape border rotate=-180, minimum height=3em, minimum width=2.5cm, drop shadow]
\tikzstyle{line} = [draw, -{>[length=2mm, width=1mm]}]
\begin{figure}[t]
	\centering
	\begin{tikzpicture}[node distance=2cm, auto]
	% Place nodes	
	\node [invisible-block] (INIT) {...};
	\node [triangle, right of=INIT, node distance=5.5cm] (FLATMAP) {\textit{flatMap}};
	\node [block, right of=FLATMAP] at (FLATMAP.right corner) (CG1) {$CG^I$};
	\node [block, right of=FLATMAP] at (FLATMAP.left corner) (CG2) {$CG^{N}$}; 
	\node [invisible-block, right of=CG1, node distance=4cm] at (CG1.north east) (SCN11) {...};
	\node [invisible-block, right of=CG1, node distance=4cm] at (CG1.20) (SCN12) {...};
	\node [invisible-block, right of=CG1, node distance=4cm] at (CG1.south east) (SCN13) {...};
	\node [invisible-block, right of=CG2, node distance=4cm] at (CG2.north east) (SCN21) {...};
	\node [invisible-block, right of=CG2, node distance=4cm] at (CG2.20) (SCN22) {...};
	\node [invisible-block, right of=CG2, node distance=4cm] at (CG2.south east) (SCN23) {...};
	
	% Draw edges
	\path [line] (INIT) -- node[]{\footnotesize $SYM\_EXPR$}(FLATMAP.apex);
	\path [line] (FLATMAP.right corner) -- node[below=1.1cm]{...} (CG1);
	\path [line] (FLATMAP.left corner) -- (CG2);
	
	\path [line] (CG1.north east) -- node[]{\footnotesize $SYM\_EXPR^{I\_I}$}(SCN11);
	\path [line] (CG1.20) -- node[below=1mm]{...} node[]{\footnotesize $SYM\_EXPR^{I\_II}$}(SCN12);
	\path [line] (CG1.south east) -- node[]{\footnotesize $SYM\_EXPR^{I\_M}$}(SCN13);
	
	\path [line] (CG2.north east) -- node[]{\footnotesize $SYM\_EXPR^{N\_I}$}(SCN21);
	\path [line] (CG2.20) -- node[below=1mm]{...} node[]{\footnotesize $SYM\_EXPR^{N\_II}$}(SCN22);
	\path [line] (CG2.south east) -- node[]{\footnotesize $SYM\_EXPR^{N\_P}$}(SCN23);
	
	
	\end{tikzpicture}
	\caption[Diagram of the \textit{flatMap} Strategy]{Diagram of the \textit{flatMap} strategy. The input parameter is initialized to the symbolic expression percolated by the previous Spark operation if there is any, otherwise, a new symbolic expression is used instead. The output value percolated to the following functions is taken from a special choice generator registered when the \textit{flatMap} transformation has finished executing. The choice generator contains all the different elements contained in the collection. Every path taken during the \textit{flatMap} transformation registers a new choice generator with all possible values returned in that particular path.}
	\label{fig:strategies:flatMap}
\end{figure}

The \textit{flatMap} transformation behaves similarly to the \textit{map} action, the only difference is that the function passed to the \textit{flatMap} has a collection of elements as an output value instead of a single element. These elements could have undergone different transformations even though they are returned in the same collection. One example of this behavior can be seen in Listing~\ref{lst:strategies:java-spark-flatmap}. Here, two different iterable collections are returned, one contains two elements resulting from two different manipulations of the initial input, while the second contains only one element with another manipulation different from the other two. Figure~\ref{fig:strategies:flatMap} depicts the symbolic execution of a \textit{flatMap} transformation according to the flatMap strategy; it shows how each path taken inside the passed function ends up in the registration of a new choice generator whose options are the possible symbolic expressions in the returned collection.

\begin{lstlisting}[
language=Java,
caption={[FlatMap Example] A trivial example of the \textit{flatMap} transformation. It shows how the returned iterable can have elements that have undergone different transformations.},
float=h,
label=lst:strategies:java-spark-flatmap
]
numbers.flatMap(t -> {				
if(t > 2) return Arrays.asList(t*2, t*3).iterator();
else return Arrays.asList(t*4).iterator();				
});
\end{lstlisting}

\textit{Pre-processing}

The pre-processing phase is identical to the pre-processing phase of the filter and map strategies. Again, the handling of the input parameters of the passed function is carried out in the same way because the passed functions have the same number of input parameters, hence their stack frame behaves the same. However, the function passed to a \textit{flatMap} transformation implements the \texttt{FlatMapFunction} interface, which differs from the \texttt{Function} interface, used by the other two transformations, by returning an the iterator of a collection instead.

\textit{Post-processing}

The post-processing phase of this strategy behaves quite differently from all the other strategies so far. The idea would be to obtain all the different symbolic expressions inside the iterable object returned by the function and use each of them to feed subsequent Spark operations. However, the problem here lies in the change of the cardinality of the possible outcomes; so far we have had always the scenario of one input producing exactly one output, but now one input could potentially imply several outputs.

For this purpose, the strategy relies on how the mocked up \textit{flatMap} transformation is implemented in \texttt{JavaRDD} class. In this implementation, the returned iterable is completely traversed using the \textit{next} method, forcing the execution to bring each element of the collection into the stack frame of the \textit{flatMap} method. This implementation does not deviate starkly from the original \textit{flatMap} operation given that in the case of the regular Spark library, the whole iterator is explored to build a new \acrshort{acr:rdd}.

Considering how the implementation behaves, the post-processing phase waits until the passed function exits and then checks for an \texttt{invokevirtual} instruction invoking the \textit{next} method. Once detected, the respective value is taken from the stack frame and added to a list of output values in the strategy. This list is filled up with every element in the iterator.

Lastly, when the \textit{flatMap} transformation exits, the strategy registers a new custom-made choice generator whose possible values are the collected output values. Registering a choice generator at this point ensures that, after backtracking, a new option from the collection will be chosen exactly at the end of the \textit{flatMap} transformation. The selected value will be used as input for any subsequent Spark operations and the analysis will continue accordingly.

The choice generator used in this scenario is called \texttt{SparkMultipleOutputChoiceGenerator}; it extends the \texttt{IntIntervalGenerator} of the \jpf{}
core. It contains a list of symbolic expressions representing the different manipulations of the input value and uses the integer range to select one of this options every time a backtrack occurs. The range is defined between zero and the size of the list minus one.
\TUsubsection{Output}
\label{subsec:module:output}

%Mention the console output of the dataset
%
%Mention the iterative datasets
%
%Mention the unfeasible path conditions

After a symbolic execution, \textit{JPF-SymSpark} produces reduced input datasets that ensure the full path coverage of the program under test. The elements in these datasets are collected during the execution and are presented to the users through the implementation of the \textit{PublisherExtension} interface provided by \jpf{}. By the means of the publisher interface, the datasets are included as part of the execution summary of \jpf{} and subsequently printed in the standard output. There are two types of datasets that can be generated as an output of the analysis depending on which strategy was chosen: the regular dataset and the iterative datasets.

\textbf{Regular dataset}

A regular dataset contains one representative element of each equivalence classes defined by the satisfiable path conditions found during the symbolic execution of the program under test. This dataset is produced when the program under test does not contain a \textit{reduce} action or the analysis was not set to conduct an iterative symbolic execution of the said operation. In such a scenario, a regular dataset is sufficient to ensure full path coverage of the program under test. The reason for this is that because no iterative analysis is required, the interrelation between the elements of the input dataset is irrelevant. 

This kind of output is sufficient when the goal of the analysis is to reason over a series of transformations. By definition, transformations in Spark have a one-to-one or one-to-many semantics (for example, \textit{filter} and \textit{flatMap} respectively) but never a many-to-one semantics as in the case of the aggregation actions. 

\clearpage

\textbf{Iterative datasets}

On the contrary, iterative datasets are produced when the analysis is set to conduct symbolic executions of iterative actions. In this case, the interrelation between the elements of the dataset is relevant to possible path conditions defined over accumulated values. For this reason, we decided to produce a dataset for each path condition found during the execution of the program under test, where each element of the dataset corresponds to a symbolic input that participated in the cumulative action. The cardinality of the dataset increases based on the number of iterations given that more iterations need larger datasets to actually be executed.

The result is presented as a family of datasets that comply with all the path conditions found during the execution. We decided to present all datasets and not only those corresponding to the final iteration because, although datasets of smaller iterations incur in some redundancy, it might be the case that the final iteration incurs in an unsatisfiable path condition for a path that was feasible in a previous iteration. Determining when a dataset is redundant is a future optimization.

Furthermore, an analysis can have both a regular dataset and iterative datasets as output. Although a program under test is analyzed following an iterative strategy, some transformations can break the control flow before reaching the aggregate action. This is the case of \textit{filter} transformation, where the negative branch is immediately backtracked after execution. In this scenario, a representative element for all possible path conditions that came to an end before reaching the iterative action is included in a regular dataset.

\textbf{Unsatisfiable path conditions}

Additionally, \textit{JPF-SymSpark} also reports those path conditions that were not satisfiable. This process gets triggered after a backtracking action resulting from the complete exploration of an execution path. If the path condition defining the explored path is satisfiable then a result is included in the respective dataset, otherwise the unsatisfiable path condition is reported directly to the standard output which in many cases is the executing console. Listing~\ref{lst:module:output:path-condition} shows the output corresponding to an unsatisfiable path condition of a sample program.

\begin{lstlisting}[
language=Bash,
caption={[Output for an Unsatisfiable Path Condition] Output for an unsatisfiable path condition. Values containing the word \texttt{SYMINT} respresent symbolic integers while values of the form \texttt{CONST\_X} are plain integer constants representing the number replacing the \texttt{X}. This path condition has a contradiction between the first and second elements of the conjunction.},
float,
label=lst:module:output:path-condition
]
Current path condition not satisfiable: constraint 3
v1_4_SYMINT > CONST_2 &&
(v1_4_SYMINT + CONST_1) <= CONST_3 &&
v1_1_SYMINT > CONST_2
\end{lstlisting}

The notification of unsatisfiable path conditions allows the users to identify control flow operations that could potentially represent incorrect or unexpected behavior. Altogether with the reduced input datasets, this notification provides the missing part for a comprehensive analysis of the symbolic execution.

