\TUchapter{Evaluation}

This chapter presents the evaluation of \textit{JPF-SymSpark} in two dimensions: a qualitative examination of the overall tool and a quantitative appraisal of iterative symbolic executions. Additionally, the chapter concludes with a discussion about the limitations of the tool and its processes. 

The qualitative examination aims to contrast \textit{JPF-SymSpark} against a series of functional requirements of an ideal symbolic execution framework for Apache Spark. The quantitative appraisal of the iterative symbolic executions explores the behavior of the iterative reduce strategy in order to highlight performance obstacles when choosing this execution approach. Lastly, the discussion on the limitations provides additional information on the constraints, for both the tool and the process, under the context of \jpf{}.

%The next sections are structured as follows: First, the requirements for an ideal symbolic execution tool for Spark programs are enunciated. Next, \textit{JPF-SymSpark} is contrasted against each of the defined requirements in order to determine if they are met. Additionally, the performance evaluation of the iterative reduce strategy is presented. Finally, the limitations of \textit{JPF-SymSpark} and the whole \acrshort{acr:jpf} framework are discussed.

\input{parts/evaluation/qualitative}
\TUsection{Quantitative Evaluation of Iterative Symbolic Executions}

This evaluation focuses on the behavior of the iterative reduce strategy. It illustrates how an increment in the number of iterations to be considered in the analysis has a direct impact in the number and size of path conditions, as well as in the performance of the constraint solvers. Moreover, it serves as an example to identify path explosion in symbolic executions.

\begin{table}[t]
	\centering
	\large
	\begin{tabular}{l||ccccc}
		& \multicolumn{5}{c}{Iterations} \\		
		& 2 & 3 & 5 & 8 & 13 \\
		\hline
		IRNC  & 0.6802 & 0.705  & 0.8558  & 1.3442   & 5.4894   \\
		IRC   & 0.5896 & 0.6251 & 15.6161 & 568.0045 & N/A      \\
		IMRNC & 0.6697 & 0.7102 & 0.9814  & 2.6217   & 21.6629  \\
		IMRC  & 0.6818 & 0.7284 & 1.1445  & 2.585    & 177.5792 \\		
	\end{tabular}	
	\caption[This]{and that.}
	\label{tab:evaluation:quantitative-time}
\end{table}

\begin{table}[t]
	\centering
	\large
	\begin{tabular}{l||cccccccccc}
		& \multicolumn{10}{c}{Iterations} \\		
		& \multicolumn{2}{c}{2} & \multicolumn{2}{c}{3} & \multicolumn{2}{c}{5} & \multicolumn{2}{c}{8} & \multicolumn{2}{c}{13} \\
		& S & U & S & U & S & U & S & U & S & U \\		
		\hline
		IRNC   & 6 & 0 & 14 & 0  & 62 & 0  & 510 & 0   & 16382 & 0               \\
		IRC    & 6 & 0 & 14 & 0  & 62 & 0  & 454 & 56  & \multicolumn{2}{c}{N/A} \\
		IMRNC  & 8 & 4 & 17 & 11 & 67 & 57 & 518 & 512 & 16395 & 16369           \\
		IMRC   & 7 & 5 & 12 & 16 & 25 & 99 & 57  & 963 & 573   & 32191           \\		
	\end{tabular}
	\caption[This]{and that.}
	\label{tab:evaluation:quantitative-path-conditions}
\end{table} 
\input{parts/evaluation/limitations}