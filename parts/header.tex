\documentclass[parskip,a4,numbers=noenddot  %,twoside
]{scrreprt}

\usepackage{tustyle}
\usepackage[main=english,ngerman]{babel}
\usepackage[utf8]{inputenc}
\usepackage{tudfonts}
\usepackage{siunitx}
\usepackage{booktabs}
\usepackage{listings}					% Creates code listings
\usepackage[markup=nocolor]{changes}	% Allows the possibility to strike text through
\usepackage[activate={true,nocompatibility},final,tracking=true,kerning=true,spacing=true,factor=1100,stretch=10,shrink=10]{microtype}
\usepackage{url}
\usepackage{blindtext}
\usepackage{setspace}
\usepackage[justification=centerlast]{caption}
\usepackage{float}
\usepackage{xcolor}

% Sets refrences so users can click and jump to the right sections
\usepackage{hyperref}

\usepackage[
	nonumberlist, 			% No display of page numbers in the lists of abbreviations and symbols
	nogroupskip, 			% No entries are grouped by initial letters and no additional vertical spaces are included when changing the initial letters
	acronym, 				% Creates a list of acronyms and abbreviations
	symbols,				% Creates a list of symbols
	toc, 					% Entry in the table of contents for the lists of abbreviations and symbols
	translate=babel 		% Translation of headings into German (if German is the main language of the document)
]{glossaries}				% Package for creating lists of abbreviations and symbols

% The bibliography is designed according to the desired options
% For MiKTeX users (Biber):
%\usepackage[backend=biber, style=numeric-comp, bibstyle=numeric, citestyle=numeric]{biblatex}
% For TeXLive users (Bibtex):
\usepackage[backend=bibtexu, style=numeric-comp, bibstyle=numeric, citestyle=numeric,firstinits=true]{biblatex}

% % % % % % % % % % % % % % % % % % % % %
%										%
%	Package Settings					%
%										%
% % % % % % % % % % % % % % % % % % % % % 

% Page margins
\geometry{top=2.5cm, left=2.5cm, right=2.5cm, bottom=2cm}			
\graphicspath{{img/}}	% Defines directories where \includegraphics looks for image files by default
\sisetup{
	exponent-product = \cdot, 			% Sets a period as a separator when using exponential
	%	output-decimal-marker = {,},		% Use only if the default is a non-english context. Sets the comma as a decimal seprator (in English is a period)
	per-mode = symbol,					% Sets fractions instead of negatives exponents
	bracket-unit-denominator = false,	% Does not surround units in the denominator with parentheses
	range-phrase = ~--~,				% Uses an en-dash for ranges whenever \SIrange or \numrange are used
	number-unit-product = \text{~},		% Fixes the distance between a number and its unit
	detect-all,							% This also takes into consideration changes in the font for numbers and units
	%	list-final-separator = ~und~,		% Use only if the default is a non-english context. Translation
	%	list-pair-separator = ~und~,		% Use only if the default is a non-english context. Translation
}

% Takes care of the coloring of internet and internal links
%\hypersetup{
%	colorlinks = true,		
%	linkcolor = red,
%	urlcolor = cyan,			% Text color for URLs
%	citecolor = green,			% Text color for citations
%	pdfborder = 1 0 1,			% Fixes the frame if one is used
%	linkbordercolor = red,		% Frame color for references (white = transparent in white background)
%	urlbordercolor = cyan,		% Frame color for URLs
%	citebordercolor = green,	% Frame color for citations (white = transparent in white backgrounds)
%	bookmarksopen = false,		% Determines whether or not the bookmark bar is expanded when the PDF is opened
%	bookmarksnumbered = false,	% Determines whether the chapter numbers are accepted into the bookmark tree
%}

% Listings default configuration
\lstset{
	basicstyle = \small\ttfamily\color{black},
	breaklines = true,
	breakatwhitespace = true,
	captionpos = b,
	commentstyle = \color{brown},
	escapeinside={/*\#}{\#*/},
	float = t,
	frame = lines,
	framexbottommargin = 10pt,
	framexleftmargin = 20pt,
	framexrightmargin = 20pt,
	framextopmargin = 10pt,
	keywordstyle = \color{purple},
	ndkeywordstyle = \color{green},
	numbers = left,
	numbersep = 10pt,
	numberstyle = \color{gray},
	rulecolor = \color{gray},
	showstringspaces = false,
	stringstyle = \color{blue},
	tabsize = 2,
	xleftmargin = .15\textwidth,
	xrightmargin = .15\textwidth
}

% Changes the name of the list of listings
\renewcommand{\lstlistlistingname}{List of \lstlistingname s}

% Sets how the captions for listings are displayed
\DeclareCaptionFormat{listing}{%
	\parbox{\textwidth-.3\textwidth+40pt}{\centerlast #1#2#3}%
}
\captionsetup[lstlisting]{format=listing,margin={.15\textwidth-20pt,0pt}}

% % % % % % % % % % % % % % % % % % % % %
%										%
% 	Bibliography Settings				%
%										%
% % % % % % % % % % % % % % % % % % % % % 

% Loads the bibliography references from the current working directory
\addbibresource{bibliography,webresources}		
\ExecuteBibliographyOptions{
	url = false,
	%	dashed=false		
	bibencoding=utf8,	% If .bib is in utf8, otherwise ascii
	bibwarn=true, 		% Warning if the bib data is corrupt
	sorting=none 		% Issues the entries in the bibliography in the order in which they were cited
}

% Bibliography settings for the german language
%\DefineBibliographyStrings{ngerman}{
%	bibliography = {Literaturverzeichnis},	% Sets the heading of the bibliography
%	urlseen      = {Zugriff\addcolon}		% Changes the label "besucht am" of URL sources to"Zugriff:"
%}

% In the bibliography, the representation of the author names follows the surname, first name format. (depending on the style, this must be adapted to get the effect, see: http://projekte.dante.de/DanteFAQ/BiblatexRegisterAutoren)
\DeclareNameAlias{default}{last-first}
% Sets the surname of the authors in capital letters
\renewcommand*{\mkbibnamelast}[1]{\textsc{#1}}		

% % % % % % % % % % % % % % % % % % % % % % % % % % % % % 
%												        %
%	New definitions and changes to existing commands	%
%														%
% % % % % % % % % % % % % % % % % % % % % % % % % % % % %

% Re-definition of the itemize environment to reduce the gaps between bullet points
\let\olditemize\itemize
\renewcommand{\itemize}{
	\olditemize
	\itemsep4pt
}

% Re-definition of the enumerate environment to reduce the gaps between numbers and content
\let\oldenumerate\enumerate
\renewcommand{\enumerate}{
	\oldenumerate
	\itemsep4pt
}

% Use only if the default is a non-english context. Replaces the abbreviation for images to "Abb." in the german language
%\addto\extrasngerman{\def\figureautorefname{Abb.}}							

% Use only if the default is a non-english context. Replaces the abbreviation for tables to "Tab." in the german language
%\addto\extrasngerman{\def\tableautorefname{Tab.}}							

% Loads the two-color identable (Uncomment if necessary)
%\makeatletter
\newcommand{\myTUDindentbar}[3][\textwidth]{%
	\setlength{\@TUD@indentbar@width}{#1}%
	\mbox{%
		\color{#2}\rule{0.67\@TUD@indentbar@width}{\@TUD@largeruleheight}%
		\color{#3}\rule{0.33\@TUD@indentbar@width}{\@TUD@largeruleheight}%
		\hspace{-\@TUD@indentbar@width}%
		\@TUD@smallrulecolor\rule[-\@TUD@rulesep-\@TUD@smallruleheight]{\@TUD@indentbar@width}{\@TUD@smallruleheight}%
	}%
}
\fancypagestyle{empty}{%
  	\fancyhf{}
  	\if@reversemargin
  	\fancyhfoffset[LO]{\marginparwidth + \marginparsep}%
  	\fancyhfoffset[RO]{0pt}%
  	\else
  	\fancyhfoffset[LO]{0pt}%
  	\fancyhfoffset[RO]{\marginparwidth + \marginparsep}%
  	\fi
  	\if@twoside
  	\if@reversemargin
  	\fancyhfoffset[LE]{0pt}%
  	\fancyhfoffset[RE]{\marginparwidth + \marginparsep}%
  	\else
  	\fancyhfoffset[LE]{\marginparwidth + \marginparsep}%
  	\fancyhfoffset[RE]{0pt}%
  	\fi
  	\fi
  	\fancyhead[C]{\myTUDindentbar[\headwidth]{tud1d}{tud9d}}
  	\fancyfoot[C]{\tudrule[\headwidth]}
  }


 \fancypagestyle{plain}{%
 	\fancyhf{}
 	\if@reversemargin
 	\fancyhfoffset[LO]{\marginparwidth + \marginparsep}%
 	\fancyhfoffset[RO]{0pt}%
 	\else
 	\fancyhfoffset[LO]{0pt}%
 	\fancyhfoffset[RO]{\marginparwidth + \marginparsep}%
 	\fi
 	\if@twoside
 	\if@reversemargin
 	\fancyhfoffset[LE]{0pt}%
 	\fancyhfoffset[RE]{\marginparwidth + \marginparsep}%
 	\else
 	\fancyhfoffset[LE]{\marginparwidth + \marginparsep}%
 	\fancyhfoffset[RE]{0pt}%
 	\fi
 	\fi
 	\ifTUD@pagingbar
 	\if@reversemargin
 	\fancyheadoffset[RO]{10mm}%
 	\else
 	\fancyheadoffset[RO]{\marginparwidth + \marginparsep + 10mm}%
 	\fi
 	\fancyhead[LO]{%
 		\pagingfont%
 		\myTUDindentbar[\headwidth - 10mm]{tud1d}{tud9d}\nobreak\hskip1.6mm\nobreak\thepage%
 	}
 	\if@twoside
 	\if@reversemargin
 	\fancyheadoffset[LE]{10mm}%
 	\else
 	\fancyheadoffset[LE]{\marginparwidth + \marginparsep + 10mm}%
 	\fi
 	\fancyhead[RE]{%
 		\pagingfont%
 		\thepage\nobreak\hskip1.6mm\nobreak\myTUDindentbar[\headwidth - 10mm]{tud1d}{tud9d}%
 	}
 	\fi
 	\else
 	\fancyhead[C]{\myTUDindentbar[\headwidth]{tud1d}{tud9d}}
 	\fi
 	\fancyfoot[C]{\tudrule[\headwidth]}
 	\ifTUD@pagingbar\else
 	\if@twoside
 	\fancyfoot[LE,RO]{\footerfont\strut\\\thepage}
 	\else
 	\fancyfoot[R]{\footerfont\strut\\\thepage}
 	\fi
 	\fi
 }
 
  %%% headings %%%
  \fancypagestyle{headings}{%
  	\fancyhf{}
  	\if@reversemargin
  	\fancyhfoffset[LO]{\marginparwidth + \marginparsep}%
  	\fancyhfoffset[RO]{0pt}%
  	\else
  	\fancyhfoffset[LO]{0pt}%
  	\fancyhfoffset[RO]{\marginparwidth + \marginparsep}%
  	\fi
  	\if@twoside
  	\if@reversemargin
  	\fancyhfoffset[LE]{0pt}%
  	\fancyhfoffset[RE]{\marginparwidth + \marginparsep}%
  	\else
  	\fancyhfoffset[LE]{\marginparwidth + \marginparsep}%
  	\fancyhfoffset[RE]{0pt}%
  	\fi
  	\fi
  	\fancyhead[C]{\myTUDindentbar[\headwidth]{tud1d}{tud9d}}
  	\fancyfoot[C]{\tudrule[\headwidth]\footerfont\strut\\\nouppercase\centermark}
  	\if@twoside
  	\fancyfoot[LE,RO]{\footerfont\strut\\\thepage}
  	\fancyfoot[RE]{\footerfont\strut\\\nouppercase\leftmark}
  	\fancyfoot[LO]{\footerfont\strut\\\nouppercase\rightmark}
  	\else
  	\fancyfoot[R]{\footerfont\strut\\\thepage}
  	\fancyfoot[L]{\footerfont\strut\\\nouppercase\rightmark}
  	\fi
  } 
 
\makeatother

% Definition of the bibeinzug environment for manually creating entries in the bibliography
\newenvironment{bibeinzug}{\hangindent=1cm}{\par}

% Definition of the bibzahl command for manually creating a numbered entry in the bibliography
\newcommand{\bibzahl}[2] {
\begin{tabular}{p{0.75cm}p{\textwidth-1.5cm}}
	[#1] &#2
\end{tabular}
}

% Definition of aliases for long, repetive commands
\newcommand{\jpf}{\acrshort{acr:jpf}~}