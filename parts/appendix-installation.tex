\TUchapter{Appendix - Installation and Use}
\label{app:installation}

This appendix aims to serve as a checklist for installing \textit{JPF-SysSpark}, as well as a brief guide on how to use the module in combination with \jpf{}.
\begin{itemize}
	\item Docker approach (preferred)
	\item Manual approach
\end{itemize}

\TUsection{Docker approach}

Docker is a widely supported platform for the creation and maintenance of virtual containers~\cite{Merkel2014}. It is designed to provide self-contained, portable environments that are ideal for distributing software with a fixed set of dependencies.

For this reason, we provide the description of a Docker container that prepares an environment with all the dependencies and configurations required by \textit{JPF-SymSpark}. This installation method is the preferred approach given that its simplicity and tested behavior. The following steps assume that Docker is already installed in the system:

\begin{enumerate}
	\item Clone the \textit{JPF-SymSpark} repository \\ \\	
		\lstinline[]|git clone https://github.com/omrsin/jpf-symbc.git|
	\item Go to the root directory of the project and build the container \\ \\
		\lstinline[]|docker build -t jpf-symspark .|
	\item Once the container is successfully built, run a container shell with \\ \\
		\lstinline[]|docker run -it jpf-symspark|
	\item Inside the container, the installation of the module can be validated by running \\ \\
		\lstinline[]|cd jpf-symspark/src/examples/de/tudarmstadt/thesis/symspark/examples/java/applied/| \\
		\lstinline[]|jpf WordCountExample.jpf|
\end{enumerate}

The output should display the outcome of the analysis run on the WordCountExample.java program.

