\TUsubsection{Instruction Factory}

% Point out how the detection of Spark operations as well relies on the combinations of the Instruction Factory, the instruction class itself and Platform-specific validator classes.

% Indicate that the only relevant instruction is INVOKEVIRTUAL

% Show a comparison between how a code looks and how its compiled bytecode instruction looks

The first step for carrying out the analysis is to determine when a Spark operation is being executed. Given that all relevant operations are implemented as non-static methods in the concrete \textit{JavaRDD} class, the bytecode instruction of interest is \textit{invokevirtual}. This instruction is in charge of dispatching Java methods, unless they are interface methods, static methods or some other special cases (\textit{invokeinterface}, \textit{invokestatic} and \textit{invokespecial} are used respectively in these cases)~\cite{Lindholm2014}.

For this purpose, we implemented the \textit{SparkSymbolicInstructionFactory} class; which extends from the \textit{SymbolicInstructionFactory} class defined in the \spf module. The goal of this class is to solely intercept calls to the \textit{invokevirtual} bytecode instruction and validate if they intend to dispatch one of the Spark operations relevant to the analysis.

Just to illustrate this situation better in the case of the Java implementation, let us assume that the \textit{filter} transformation is being called on an existing \acrshort{acr:rdd} such as

\hspace*{1cm} \lstinline[language=Java,]|rdd.filter(...)|

then, the corresponding bytecode will look like the following

\hspace*{1cm} \lstinline[]|invokevirtual PATH/JavaRDD.filter:(PATH/Function;) PATH/JavaRDD;|

with ``PATH'' representing the full package path where the classes or interfaces are located. The rest of the instruction represents the method name and the method descriptor; sufficient information for identifying the methods of interest. The function parameter was intentionally omitted because, although the passed parameter must implement the \textit{Function} interface, this can be done in several ways; being the most common lambda functions and anonymous classes. At this point, neither of this two approaches represent a difference when detecting the Spark operation, however, it will require special attention later on when detecting the invocation of the passed function.

